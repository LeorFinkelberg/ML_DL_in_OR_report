\documentclass[%
	11pt,
	a4paper,
	utf8,
	%twocolumn
		]{article}	

\usepackage{style_packages/podvoyskiy_article_extended}


\begin{document}
\title{Пояснительная записка\\{\large Вычислительные техники решения задач линейного программирования в частично-целочисленной постановке и приемы работы с решателем SCIP}}

\author{\itshape Подвойский А.О., Глазунова Е.В.}


\date{}
\maketitle

\thispagestyle{fancy}

%Здесь приводятся заметки по специальным вопросам теории оптимизации

%\shorttableofcontents{Краткое содержание}{1}

\tableofcontents

\section{Ключевые термины и определения}

\emph{Сценарий} -- это математическая постановка задачи, описанная в терманах математического программирования (например, линейного)

\emph{Сценарий обучающего поднабора} -- это сценарий из коллекции сценариев, которые используются на {обучающей фазе} алгоритма машинного обучения

\emph{Сценарий тестового поднабора} -- это сценарий, который используется {для построения прогноза} с помощью алгоритма машинного обучения

\section{Ключевые компоненты платформы SCIP}

\subsection{Решатель SCIP. Общие сведения}

SCIP (Solving Constraint Integer Programs) \url{https://www.scipopt.org/} -- решатель, предназначенный для решения задач \emph{линейного} и \emph{нелинейного} программирования в частично-целочисленной постановке.

\subsubsection{Установка решателя SCIP}
Решатель проще всего установить вместе с оберткой PySCIPOpt \url{https://github.com/scipopt/PySCIPOpt} с помощью менеджеров \texttt{pip} или \texttt{conda}
\begin{lstlisting}[
style = bash,
numbers = none
]
# установить последнюю доступную версию SCIP
$ pip install pyscipopt
$ conda install -c conda-forge pyscipopt
# установить заданную версию SCIP
$ conda install -c conda-forge pyscipopt=3.4.0
\end{lstlisting}

\subsubsection{Приемы работы с решателем SCIP в интерактивной оболочке \texttt{scip}}

Для того чтобы сделать логи более подробными следует включить следующие строки в конфигурационный файл SCIP
\begin{lstlisting}[
title = {\sffamily scip.set},
style = bash,
numbers = none
]
...
display/lpinfo = TRUE
display/ninfeasleaves/active = 2
display/allviols = TRUE
\end{lstlisting}

\subsubsection{Приемы работы с решателем SCIP через обертку PySCIPOpt}

Работа над задачей начинается с создания пустого экземпляра модели
\begin{lstlisting}[
style = ironpython,
numbers = none
]
import pyscipopt

model = pyscipopt.Model()
\end{lstlisting}

На созданном экземпляре можно вызывать методы чтения модели, конфигурационного файла параметров решателя и т.д.
\begin{lstlisting}[
style = ironpython,
numbers = none,
]
model.readProblem("./problem.lp")
model.readParams("./scip.set")
...
\end{lstlisting}



\subsection{Декомпозиционный решатель GCG. Общие сведения}

GCG \url{https://gcg.or.rwth-aachen.de/#about} -- это универсальный декомпозиционный решатель для задач линейного программирования в частично-целочисленной постановке, расширающий возможности базового решателя SCIP.

Он выявляет структуры в модели, к которым могут быть применены \emph{переформулировка Данцига-Вольфе} или \emph{декомпозиция Бендера}.

Модфицированная постановка задачи (после переформулировки Данцига-Вольфе) решается с помощью обобщения метода ветвей-и-границ, а именно с помощью метода ветвей-штрафов-секущих (branch-price-and-cut), включающего различные механизмы поиска решения -- превичные эвристики, стратегии ветвления, стратегии стабилизации, стратегии назначения штрафов и пр.

\subsubsection{Установка решетеля GCG}

Проще всего решатель установить вместе с обреткой PyGCGOpt \url{https://github.com/scipopt/PyGCGOpt} с помощью мендежера пакетов \texttt{conda}
\begin{lstlisting}[
style = bash,
numbers = none
]
$ conda install -c conda-forge pygcgopt
\end{lstlisting}



\subsubsection{Приемы работы с решателем GCG в интерактивной оболочке \texttt{gcg}}

Прочитать постановку задачи
\begin{lstlisting}[
style = bash,
numbers = none
]
GCG> read problem.lp
\end{lstlisting}

Запустить процедуру редуцированния размерности
\begin{lstlisting}[
style = bash,
numbers = none
]
GCG> presolve
\end{lstlisting}

Запустить процедуру поиска структур в матрице ограничений
\begin{lstlisting}[
style = bash,
numbers = none
]
GCG> detect
\end{lstlisting}

Записать постановку задачи сниженной размерности для \texttt{gnuplot}
\begin{lstlisting}[
style = bash,
numbers = none
]
GCG> write problem problem_reduced.gp
\end{lstlisting}

Фрагмент gp-файла
\begin{lstlisting}[
style = bash,
numbers = none
]
set encoding utf8
set terminal pdf
set output "problem_reduced.pdf"
set xrange [-1:506441]
set yrange[347788:-1]
set object 1 rect from 0,0 to 506441,183384 fc rgb "#1340C7"
set object 3 rect from 163304,183384 to 163306,183385 fc rgb "#718CDB" 
set object 4 rect from 163306,183385 to 163308,183386 fc rgb "#718CDB" 
set object 5 rect from 163308,183386 to 163310,183387 fc rgb "#718CDB" 
set object 6 rect from 163310,183387 to 163312,183388 fc rgb "#718CDB" 
set object 7 rect from 163312,183388 to 163314,183389 fc rgb "#718CDB" 
set object 8 rect from 163314,183389 to 163316,183390 fc rgb "#718CDB" 
set object 9 rect from 163316,183390 to 163318,183391 fc rgb "#718CDB" 
set object 10 rect from 163318,183391 to 163320,183392 fc rgb "#718CDB" 
set object 11 rect from 163320,183392 to 163322,183393 fc rgb "#718CDB" 
...
\end{lstlisting}

Создать pdf-файл декомпозиции задачи после шага снижения размерности
\begin{lstlisting}[
style = bash,
numbers = none
]
$ gnuplot problem_reduced.gp
\end{lstlisting}

\subsubsection{Приемы работы с решателем GCG через обертку PyGCGOpt}

\section{Выявленные баги SCIP и тонкости процедуры поиска решения}

\subsection{Недопустимое решение для релаксированной постановки задачи}

По состоянию на 18.06.2022 г. решатель SCIP версии 8.0.0 с оберткой PySCIPOpt версий 4.0.0 и 4.2.0 для операционной системы Windows 10 \emph{релаксированную постановку задачи} (т.е. при снятых ограничениях на целочисленность переменных) оценивает как неспособную привести к допустимому решению.

SCIP версии 7.0.3 (PySCIPOpt 3.4.0) как на операционной системе Windows 10, так и на Unix-подобных операционных системах (в частности, MacOS Monterey 12.1 и Linux Centos 7) решает задачу в релаксированной постановке корректно.

\subsection{Неединственность релаксированного решения}

Если эвристические приемы строятся на базе релаксированного решения задачи, важно помнить, что релаксированные решения, полученные с помощью различных решателей с точки зрения распределения значений переменных могут существенно различаться\footnote{Потому как гиперплоскость целевой функции может касаться политопа не в вершине, а по грани}, не смотря на то, что во всех случах зазор будет нулевым и целевая функция будет имееть одно и тоже значение (с оговоркой на допуск точности решателя). 

\subsection{Замечание о стабильности работы решателя SCIP на различных операционных системах}

\begin{itemize}
	\item Вычислительные эксперименты проводились на трех версиях решателя SCIP (7.0.0, 7.0.3, 8.0.0) и трех платформах: Windows~10, MacOS (Monterey~12), Linux (Centos~7). Разброс времени поиска решения для каждой конфигурации решателя оценивается минимум по 3 запускам сценария
	
	\item На текущий момент наиболее стабильные и наиболее адекватные результаты получаются
	\begin{itemize}
		\item для ОС Linux (Centos 7) и ОС MacOS (Monterey12) на решателе SCIP версии 7.0.3 (обертка PySCIPOpt 3.4.0) и платформе Ecole версии 0.7.3 , собранных для однопоточной реализации
		
		\item для ОС Windows 10 на решателе SCIP версии 8.0.0 (обертка PySCIPOpt 4.0.0), собранном для однопоточной реализации
	\end{itemize}
	
	\item Последняя доступная версия решателя SCIP 8.0.0 (PySCIPOpt 4.1.0) на MacOS (Monterey 12.1) и Linux (Centos 7) при тех же настройках, что и для SCIP версии 7.0.3, как правило, работает значительно медленнее (2.5-2.85 раза) и в большинстве случаев либо не успевает найти решение за отведенное время, либо «просаживает» целевую функцию
\end{itemize}

\section{Альтернативные решатели с открытым исходным кодом}

\subsection{Решатель HIGHS}

\subsubsection{Установка решателя на Centos 7}

Установить решатель HIGHS \url{https://ergo-code.github.io/HiGHS/get-started.html} можно следующим образом
\begin{enumerate}
	\item Скачать репозиторий проекта
\begin{lstlisting}[
style = bash,
numbers = none
]
$ git clone https://github.com/ERGO-Code/HiGHS.git
\end{lstlisting}
    \item Установить \texttt{cmake} версии \texttt{>=3.15}
\begin{lstlisting}[
style = bash,
numbers = none
]
# https://cmake.org/download/
$ wget https://github.com/Kitware/CMake/releases/download/v3.24.2/cmake-3.24.2.tar.gz
$ tar -xvf cmake-3.24.2.tar.gz
$ cd cmake-3.24.2
$ ./bootstrap --prefix=/usr --datadir=share/cmake --docdir=doc/cmake && make
$ sudo make install
$ cmake --version  # cmake version 3.24.2
\end{lstlisting}
    \item Установить альтернативную версию комилятора \texttt{gcc} (например, версии 7) для сборки проекта
\begin{lstlisting}[
style = bash,
numbers = none	
]
# https://linuxize.com/post/how-to-install-gcc-compiler-on-centos-7/
$ gcc --version  # gcc (GCC) 4.8.5 20150623 (Red Hat 4.8.5-36)
\end{lstlisting}

Чтобы получить доступ к альтернативной версии компилятора GCC 7, требуется запустить новый сеанс командной оболочки с помощью утилиты \texttt{scl}
\begin{lstlisting}[
style = bash,
numbers = none
]
$ scl enable devtoolset-7 bash
# или для ZSH
# scl enable devtoolset-7 zsh
$ gcc --version  # gcc (GCC) 7.3.1 20180303 (Red Hat 7.3.1-5)
\end{lstlisting}

    \item В директории проекта \texttt{HIGHS} создать поддиректорию \texttt{build} и запустить из-под нее утилиту \texttt{cmake}
\begin{lstlisting}[
style = bash,
numbers = none
]
$ cd HiGHS
$ mkdir build
$ cd build
$ cmake -DFAST_BUILD=ON ..
$ cmake --build .
# Чтобы убедится в том, что сборка прошла успешно, рекомендуется запустить быстрые тесты
$ ctest
\end{lstlisting}

В результате будет создан исполняемый файл \texttt{build/bin/highs}
    
    \item Добавить путь до утилиты в конфигурационный файл оболочки
\begin{lstlisting}[
title = {\sffamily .zshrc},
style = bash,
numbers = none
]
...
export PATH=${HOME}/Projects/HiGHS/build/bin:$PATH
\end{lstlisting}

После внесения изменений в конфигурационный файл, можно перечитать конфигурацию сессии
\begin{lstlisting}[
style = bash,
numbers = none
]
$ source .zshrc
\end{lstlisting}
\end{enumerate}
\vspace*{3mm}

\subsubsection{Приемы работы с решателем}

Для запуска решателя в MILP-режиме требуется только передать путь до \texttt{*.lp/*.mps}-файла
\begin{lstlisting}[
style = bash,
numbers = none
]
$ highs /path/to/model.lp
$ highs --help
\end{lstlisting}

Для запуска решателя в режиме поиска релаксированного решения требуется параметру \verb|--solver| передать название метода (на текущий момент поддерживается только симплекс-метод \texttt{simplex} и метод внутренней точки \texttt{ipm})
\begin{lstlisting}[
style = bash,
numbers = none
]
# LP-задача будет решаться методом внутренней точки
$ highs --solver ipm --model_file 50197df7_bin.lp
\end{lstlisting} 

Запуск решателя в параллельном MILP-режиме, с шагом снижения размерности задачи и ограничением по времени расчета будет выглядеть так
\begin{lstlisting}[
style = bash,
numbers = none
]
$ highs \
        --model_file 514.lp \
        --presolve on \
        --parallel on \
        --time_limit 950 \ 
        --solution_file highs_514.sol
\end{lstlisting}

Список управляющих параметров решателя доступен на странице документации HiGHS для интерфейса Rust \url{https://www.maths.ed.ac.uk/hall/HiGHS/HighsOptions.html}
\begin{lstlisting}[
style = bash,
numbers = none
]
HiGHS Options
presolve
Presolve option: "off", "choose" or "on"
type: string, advanced: false, default: "choose"
solver
Solver option: "simplex", "choose" or "ipm"
type: string, advanced: false, default: "choose"
parallel
Parallel option: "off", "choose" or "on"
type: string, advanced: false, default: "choose"
time_limit
Time limit
type: double, advanced: false, range: [0, inf], default: inf
ranging
Compute cost, bound, RHS and basic solution ranging: "off" or "on"
type: string, advanced: false, default: "off"
infinite_cost
Limit on cost coefficient: values larger than this will be treated as infinite
type: double, advanced: false, range: [1e+15, inf], default: 1e+20
infinite_bound
Limit on |constraint bound|: values larger than this will be treated as infinite
type: double, advanced: false, range: [1e+15, inf], default: 1e+20
small_matrix_value
Lower limit on |matrix entries|: values smaller than this will be treated as zero
type: double, advanced: false, range: [1e-12, inf], default: 1e-09
large_matrix_value
Upper limit on |matrix entries|: values larger than this will be treated as infinite
type: double, advanced: false, range: [1, inf], default: 1e+15
primal_feasibility_tolerance
Primal feasibility tolerance
type: double, advanced: false, range: [1e-10, inf], default: 1e-07
dual_feasibility_tolerance
Dual feasibility tolerance
type: double, advanced: false, range: [1e-10, inf], default: 1e-07
ipm_optimality_tolerance
IPM optimality tolerance
type: double, advanced: false, range: [1e-12, inf], default: 1e-08
objective_bound
Objective bound for termination
type: double, advanced: false, range: [-inf, inf], default: inf
objective_target
Objective target for termination
type: double, advanced: false, range: [-inf, inf], default: -inf
random_seed
random seed used in HiGHS
type: HighsInt, advanced: false, range: {0, 2147483647}, default: 0
threads
number of threads used by HiGHS (0: automatic)
type: HighsInt, advanced: false, range: {0, 2147483647}, default: 0
highs_debug_level
Debugging level in HiGHS
type: HighsInt, advanced: false, range: {0, 3}, default: 0
highs_analysis_level
Analysis level in HiGHS
type: HighsInt, advanced: false, range: {0, 63}, default: 0
simplex_strategy
Strategy for simplex solver
type: HighsInt, advanced: false, range: {0, 4}, default: 1
simplex_scale_strategy
Simplex scaling strategy: off / choose / equilibration / forced equilibration / max value 0 / max value 1 (0/1/2/3/4/5)
type: HighsInt, advanced: false, range: {0, 5}, default: 1
simplex_crash_strategy
Strategy for simplex crash: off / LTSSF / Bixby (0/1/2)
type: HighsInt, advanced: false, range: {0, 9}, default: 0
simplex_dual_edge_weight_strategy
Strategy for simplex dual edge weights: Choose / Dantzig / Devex / Steepest Edge (-1/0/1/2)
type: HighsInt, advanced: false, range: {-1, 3}, default: -1
simplex_primal_edge_weight_strategy
Strategy for simplex primal edge weights: Choose / Dantzig / Devex (-1/0/1)
type: HighsInt, advanced: false, range: {-1, 1}, default: -1
simplex_iteration_limit
Iteration limit for simplex solver
type: HighsInt, advanced: false, range: {0, 2147483647}, default: 2147483647
simplex_update_limit
Limit on the number of simplex UPDATE operations
type: HighsInt, advanced: false, range: {0, 2147483647}, default: 5000
ipm_iteration_limit
Iteration limit for IPM solver
type: HighsInt, advanced: false, range: {0, 2147483647}, default: 2147483647
simplex_min_concurrency
Minimum level of concurrency in parallel simplex
type: HighsInt, advanced: false, range: {1, 8}, default: 1
simplex_max_concurrency
Maximum level of concurrency in parallel simplex
type: HighsInt, advanced: false, range: {1, 8}, default: 8
output_flag
Enables or disables solver output
type: bool, advanced: false, range: {false, true}, default: true
log_to_console
Enables or disables console logging
type: bool, advanced: false, range: {false, true}, default: true
solution_file
Solution file
type: string, advanced: false, default: ""
log_file
Log file
type: string, advanced: false, default: "Highs.log"
write_solution_to_file
Write the primal and dual solution to a file
type: bool, advanced: false, range: {false, true}, default: false
write_solution_style
Write the solution in style: 0=>Raw (computer-readable); 1=>Pretty (human-readable)
type: HighsInt, advanced: false, range: {0, 1}, default: 0
mip_detect_symmetry
Whether symmetry should be detected
type: bool, advanced: false, range: {false, true}, default: true
mip_max_nodes
MIP solver max number of nodes
type: HighsInt, advanced: false, range: {0, 2147483647}, default: 2147483647
mip_max_stall_nodes
MIP solver max number of nodes where estimate is above cutoff bound
type: HighsInt, advanced: false, range: {0, 2147483647}, default: 2147483647
mip_max_leaves
MIP solver max number of leave nodes
type: HighsInt, advanced: false, range: {0, 2147483647}, default: 2147483647
mip_lp_age_limit
maximal age of dynamic LP rows before they are removed from the LP relaxation
type: HighsInt, advanced: false, range: {0, 32767}, default: 10
mip_pool_age_limit
maximal age of rows in the cutpool before they are deleted
type: HighsInt, advanced: false, range: {0, 1000}, default: 30
mip_pool_soft_limit
soft limit on the number of rows in the cutpool for dynamic age adjustment
type: HighsInt, advanced: false, range: {1, 2147483647}, default: 10000
mip_pscost_minreliable
minimal number of observations before pseudo costs are considered reliable
type: HighsInt, advanced: false, range: {0, 2147483647}, default: 8
mip_report_level
MIP solver reporting level
type: HighsInt, advanced: false, range: {0, 2}, default: 1
mip_feasibility_tolerance
MIP feasibility tolerance
type: double, advanced: false, range: [1e-10, inf], default: 1e-06
mip_heuristic_effort
effort spent for MIP heuristics
type: double, advanced: false, range: [0, 1], default: 0.05
\end{lstlisting}

{\noindent\color{red}TODO: При запуске решатля в режиме поиска релаксированного решения процесс зависает, если заданы параметры, управляющие подробностью вывода
\begin{itemize}
	\item \texttt{highs\_debug\_level = 2},
	
	\item  \texttt{mip\_report\_level = 2}
\end{itemize}
}

Задать значения управлющих параметров можно в set-файле HiGHS
\begin{lstlisting}[
title = {\sffamily highs.set},
style = bash,
numbers = none
]
time_limit = 7200
simplex_iteration_limit = 10000
ipm_iteration_limit = 5000
...
\end{lstlisting}

Теперь для запуска решателя в специфицрованном режиме остается только передать путь до файла настроек параметру \verb|--options_file|
\begin{lstlisting}[
style = bash,
numbers = none	
]
$ highs --model_file 496.lpl --options_file highs.set
\end{lstlisting}

\subsection{Решатель OPTIMUS (Scala)}


\section{Сжатая сводка результатов вычислительных экспериментов}

{
Все эксперименты проводились на ОС Linux Centos 7 Intel Core™ i7 (8 CPUs), 3.6GHz; RAM 32Gb. Использовался MILP-решатель SCIP 7.0.3 (Python-обертка PySCIPOpt 3.4.0) и Python~3.8.0.
}

Развернутая сводка результатов приводится по ссылке \url{https://docs.google.com/document/d/16p8_VjZaHCBdDWo_YNZaEpZVFgmLyDi5A61O4gX3zK8/edit?usp=sharing}

{Обозначения}
\begin{itemize}
	\item CBC+DOH – доменно-ориентированные эвристики, работающие поверх решателя CBC.
	
    \item CBC+MS - мера подобия релаксированного решения, работающая поверх решателя CBC.
    
    \item SCIP(d) – решатель SCIP с настройками по умолчанию.
    
    \item SUH – метаконфигурация, работающая поверх решателя SCIP: подавляется подгруппа первичных эвристик низкой эффективности.
    
    \item FZBIVSUHPB – метаконфигурация, работающая поверх решателя SCIP: подавляется подгруппа первичных эвристик низкой эффективности; при ветвлении и разрешении конфликтов решатель отдает предпочтение бинарными переменным; фиксируются нулевые бинарные и целочисленные переменные релаксированного решения.
    
    \item {EAD(contamination; file\_name)} – модель машинного обучения, работающая поверх решателя SCIP: подавляется подгруппа первичных эвристик низкой эффективности; при ветвлении и разрешении конфликтов решатель отдает предпочтение бинарными переменным; частично-заданное решение на фиксациях строится на основании прогноза ансамбля детекторов аномалий; {contamintion} – доля аномальных экземпляров в наборе данных, {file\_name} – имя lp-файла математической постановки, на котором обучался ансамбль детекторов аномалий.
    
    \item Detector\_name(contamination; file\_name) – детектор аномалий, работающий поверх решателя SCIP: подавляется подгруппа первичных эвристик низкой эффективности; при ветвлении и разрешении конфликтов решатель отдает предпочтение бинарными переменным; частично-заданное решение на фиксациях строится на основании прогноза детектора аномалий; contamintion – доля аномальных экземпляров в наборе данных, file\_name – имя lp-файла математической постановки, на котором обучался детектор аномалий.
    
    \item RELAX - релаксированное решение, найденное с помощью решателя SCIP.
\end{itemize}

\textbf{Выводы}
\begin{enumerate}
	\item На всех сценариях группы ИКП метаконфигурация FZBIVSUHPB помогает решателю SCIP найти более низкое значение целевой функции и за меньшее время.
	
	\item На всех сценариях группы ИКП (за исключением сценариев \texttt{a78cbead\_bin.lp}, \texttt{7fac4231\_bin.lp} и \texttt{50197df7\_bin.lp}) ансамбль детекторов аномалий без подбора параметра контаминации EAD(0.10; \texttt{f398266b\_bin.lp})  помогает решателю SCIP найти более низкое значение целевой функции и за меньшее время. На сценариях \texttt{a78cbead\_bin.lp}, \texttt{7fac4231\_bin.lp} и \texttt{50197df7\_bin.lp} прием EAD не смог найти решение за отведенное время.
	
	\item На сценариях \texttt{514.lp}, \texttt{519.lp}, \texttt{a78cbead\_bin.lp}, \texttt{7fac4231\_bin.lp} и \texttt{50197df7\_bin.lp} изолированные детекторы аномалий помогают решателю SCIP найти более низкое значение целевой функции и за меньшее время.
	
	\item На сценариях \texttt{514.lp}, \texttt{519.lp}, \texttt{a78cbead\_bin.lp}, \texttt{7fac4231\_bin.lp} и \texttt{50197df7\_bin.lp} изолированные детекторы аномалий находят решения, которые по сравнению с решениями, полученными средствами CBC+DOH(MS), оказываются лучше в среднем на 50.73\% по временным издержкам и в среднем на 6.32\% – по целевой функции.
	
	\item На всех сценариях группы ИКП (за исключением сценариев \texttt{514.lp} и \texttt{519.lp}) метаконфигурация FZBIVSUHPB находит решения, которые оказываются нехуже решений, полученных с помощью CBC+DOH(MS), как с точки зрения полного времени расчета (среднее улучшение 62.16\%), так и с точки зрения целевой функции (среднее улучшение 7.03\%). На сценарии \texttt{514.lp} метаконфигурация получает решение, которое только по целевой функции (+18.616\%) превосходит решение, найденное средствами CBC+DOH(MS). На сценарии \texttt{519.lp} решение метаконфигурации уступает решению, найденному с помощью CBC+DOH(MS) и по временным издержкам (-14.29\%) и по целевой функции (-2.302\%).
\end{enumerate}

\begin{figure}[!h]
	\centering
	\includegraphics[scale=0.57]{figures/summary_FZB_EAD_for_IKP.pdf}
	\caption{ Сводка результатов вычислительных экспериментов на сценариях группы ИКП }\label{fig:summary_FZB_EAD_for_IKP}
\end{figure}









\section{Приемы поиска решения}

\subsection{Прием фиксации бинарно-целочисленных переменных в релаксированном решении}\label{sec:bin_int_relax_fix}

Часто фиксация целочисленных переменных\footnote{Вообще говоря, фиксировать можно не только бинарные и целочисленные переменные} в релаксированном решении приводит к приемлемому допустимому целочисленному решению, которое потом можно использовать как <<теплый старат>> или как базовое решение для других схем фиксации.
\begin{lstlisting}[
style = ironpython,
numbers = none
]
ZERO = 0.0
...
relax_sol: pd.Series = read_relax_sol(path_to_relax_sol)

model = pyscipopt.Model()
model.readProblem(path_to_lp_file)
model.readParams(path_to_set_file)

all_vars: t.List[pyscipopt.scip.Variable] = model.getVars()
bin_vars: t.List[pyscipopt.scip.Variable] = extract_vars_set_type(all_vars, BINARY)
int_vars: t.List[pyscipopt.scip.Variable] = extract_vars_set_type(all_vars, INTEGER)

all_zero_bin_vars: t.List[
	pyscipopt.scip.Variable
] = extract_from_relax_sol_zero_vars(
	relax_sol,
	sub_group_vars=bin_vars,
)
all_zero_int_vars: t.List[
	pyscipopt.scip.Variable
] = extract_from_relax_sol_zero_vars(
	relax_sol,
	sub_group_vars=int_vars,
)

for var in all_zero_bin_vars + all_zero_int_vars:
	model.fixVar(var, ZERO)

model.optimize()
...
\end{lstlisting}


\subsection{Прием подавления подгруппы первичных эвристик низкой эффективности}\label{sec:suh}

В некоторых случаях отдельные первичные эвристики могут оказаться не способными справится со своей задачей, не оказывая никакого влияния на процедуру поиска решения, и все же потреблять предоставленные ресурсы.

Такие эвристики -- условимся их называть первичными эвристиками низкой эффективности (ПЭНЭ) -- можно выявить путем анализа статистической сводки \verb|stat|-файла в разделе Primal Heuristics
\begin{lstlisting}[
title = {\sffamily Фрагмент файла статистической сводки 337\_bin\_default.stat},
style = bash,
numbers = none
]
...
Primal Heuristics  :   ExecTime  SetupTime      Calls      Found       Best
  LP solutions     :       0.00          -          -          0          0
	relax solutions  :       0.00          -          -          0          0
	pseudo solutions :       0.00          -          -          0          0
	...
	conflictdiving   :       0.00       0.00          0          0          0
	crossover        :       0.00       0.00          0          0          0
	dins             :       0.00       0.00          0          0          0
	distributiondivin:       0.00       0.00          0          0          0
	dualval          :       0.00       0.00          0          0          0
	farkasdiving     :    2032.89       0.00          1          0          0  # <- NB
	feaspump         :     882.12       0.00          1          0          0  # <- NB
	fixandinfer      :       0.00       0.00          0          0          0
	...
	intdiving        :       0.00       0.00          0          0          0
	intshifting      :      52.99       0.00          1          1          1
	...
\end{lstlisting}

В данном случае ПЭНЭ являются \texttt{farkasdiving} и \texttt{feaspump}. Чтобы подавить эти эвристики при следующем запуске \texttt{SCIP}, достаточно включить следующие строки в конфигурационный файл \texttt{scip.set}\footnote{При запуске интерактивной сесии через утилиту командной строки \texttt{scip}, решатель ищет этот файл в текущей директории и, если находит, автоматически вычитывает. При работе через PySCIPOpt требуется явно передавать путь до файла методу модели \texttt{readParams()}}
\begin{lstlisting}[
title = {\sffamily scip.set},
style = bash,
numbers = none
]
...
heuristics/farkasdiving/freq = -1
heuristics/feaspump/freq = -1
...
\end{lstlisting}

Доступ к статистической сводке можно получить либо в сессии \texttt{SCIP}, либо через одну из оберток над решателем (например, с помощью PySCIPOpt)
\begin{lstlisting}[
title = {\sffamily Фрагмент сессии scip. Получение статистической сводки},
style = bash,
numbers = none
]
...
SCIP> read file.lp
SCIP> opt
SCIP> display stat
\end{lstlisting}

\begin{lstlisting}[
title = {\sffamily Получение статистической сводки через обертку PySCIPOpt},
style = ironpython,
numbers = none
]
import pyscipopt

model = pyscipopt.Model()
model.readProblem("...")
model.readParams("...")
model.optimize()

model.printStatistics()
\end{lstlisting}





\subsection{Прием подбора порога бинаризации для бинарных переменных в релаксированном решении}\label{sec:find_bin_thresh}

Условимся \emph{фиксацией} называть стратегию инициализации подгруппы переменных $ x_k $ (вещественных, бинарных или целочисленных), значения которых задаются на основе каких-либо эврестических соображений, например, касающихся специальных свойств матрицы ограничений, и способных в результате привести к такой постановке задачи, которую, используя механизмы первичных эвристик, сепараторов, пропагаторов и пр. можно развить в \emph{допустимое целочисленное решение}.

Базовая идея построения \emph{фиксации на бинарных переменных} заключается в том, чтобы значения бинарных переменных в релаксированном решении\footnote{Верхний левый индекс <<$ r $>> указывает на релаксированное значение, а верхний правый <<$ (b) $>> -- на то, что речь идет о бинарной переменной} $ {\{{}^rx^{(b)}_k\}}_{k=1, \ldots} $ интерпретировать как \emph{степень уверенности} решателя в том, что рассматриваемую бинарную переменную можно выставить в единицу.

Если значение $ k $-ой бинарной переменной $ {}^rx_k^{(b)}$ превосходит некоторый \emph{порог}~$ \theta $, то переменная выставляется в единицу, в противном случае -- в ноль. Порог подбирается итерационно, начиная с некоторого нижнего значения $ \theta_l $ (по умолчанию $ \theta_l = 0 $), увеличивая текущее значение порога на величину шага $ \Delta \theta $ и заканчивая верхним значением порога $ \theta_u $ (по умолчанию $ \theta_u = 1 $).

Для практических целей достаточно остановится на наименьшем значении порога $ \theta $, который отвечает такой фиксации, которую решатель SCIP не отклоняет как неспособную привести к допустимому целочисленному решению.
\begin{lstlisting}[
title = {\sffamily Фрагмент лога решателя SCIP для случая фиксации, которую невозможно развить\\ в допустимое целочисленное решение},
style = bash,
numbers = none	
]
...
SCIP Status        : problem is solved [infeasible]
Solving Time (sec) : 3.00
Solving Nodes      : 0
Primal Bound       : +1.00000000000000e+20 (0 solutions)
Dual Bound         : +1.00000000000000e+20
Gap                : 0.00 %
original problem has 740251 variables (2666 bin, 147789 int, 0 impl, 589796 cont) and 545350 constraints
...
\end{lstlisting}

После того как порог $ \theta $ подобран, бинарные переменные разбиваются на две подгруппы: подгруппу бинарных переменных, выставленных в ноль $ \{x_k^{(b_0)}\} $, и подгруппу бинарных переменных, выставленных в единицу $ \{ x_k^{(b_1)} \} $. Долю бинарных переменных, выставленных в ноль обозначим через~$ \delta_{b_0} $, долю бинарных переменных, выставленных в единицу -- через $ \delta_{b_1} $, а целевую функцию, найденную при заданных долях -- через $ f_{\theta}(\delta_{b_0}, \delta_{b_1}) $.

В результате получаем исследовательский инструмент, который дает возможность управлять решением через подбор долей $ \delta_{b_0} $ и $ \delta_{b_1} $ при найденном пороге $ \theta $. Часто оказывается эффективным прием управления решением через подбор доли нулевых бинарных переменных $ \delta_{b_0} $.

Целевая функция, вычисленная при единичной доле нулевых бинарных переменных $ f_{\theta}(\delta_{b_0}=1) $, как правило, значительно уступает целевой функции релаксированного решения $ f_r $. Но тем неменее это решение может быть улучшено, сокращением доли $ \delta_{b_0} $ (см.~\pic{fig:a78cbeadfracbinzeros} и \pic{fig:337fracbinzeros}).

\begin{figure}[!h]
	\centering
	\includegraphics[scale=0.45]{figures/a78cbead_frac_bin_zeros.pdf}
	\caption{ Зависимость верхней границы решения от доли бинарных переменных, \\выставленных в ноль. Сценарий \texttt{a78cbead} }\label{fig:a78cbeadfracbinzeros}
\end{figure}

\begin{figure}[!h]
	\centering
	\includegraphics[scale=0.45]{figures/337_frac_bin_zeros.pdf}
	\caption{ Зависимость верхней границы решения от доли бинарных переменных, \\выставленных в ноль. Сценарий \texttt{337} }\label{fig:337fracbinzeros}
\end{figure}

Как видно из графиков, на кривой изменения верхней границы решения существует точка с наименьшим значением целевой функции $ f_{\theta}(\delta_{b_0}) $ допустимого целочисленного решения. Эта точка и будет <<оптимальной>> для рассматриваемого сценария.

\section{Методы машинного обучения в задачах комбинаторной оптимизации}

\subsection{Постановка задачи}

Цель: Разработать процедуру построения частично-заданного решения на фиксациях для сценариев с матрицей ограничений произольной структуры.

Вход: произвольная матрица ограничений\footnote{Предполагается, что матрица ограничений имеет низкую меру обусловленности}.

Выход: набор бинарных и целочисленных переменных, фиксация которых в ноль с высокой вероятностью приведет к допустимому целочисленному решению.

База: частично-заданное решение, построенное на фиксациях нулевых бинарных и целочисленных переменных в релаксированном решении.

\subsection{Концепт матрицы признакового описания бинарных и целочисленных переменных}

В качестве признаков бинарно-целочисленных переменных предлагается использовать:
\begin{enumerate}
	\item {\color{deepgreen} \itshape важный признак} Значение переменной $ x_i $ в <<усредненном>> релаксированном решении\footnote{Задача линейного программирования в релаксированной постановке решается с использованием различных методов (двойственный симплекс-метод, метод внутренней точки и т.д.), а затем полученные решения усредняются},
	
	\item Модифицированную Z-оценку на <<усредненном>> релаксированном решении,
	
	\item {\color{red} \itshape бесполезный признак} Дробную часть значения переменной $ x_i $ в <<усредненном>> релаксированном решении,
	
	\item {\color{deepgreen} \itshape важный признак} Пороги бинаризации на <<усредненном>> релаксированном решении (каждый порог это отдельный принак),
	
	\item {\color{deepgreen} \itshape важный признак} Число ограничений $ n_i $, в которые входит рассматриваемая переменная $ x_i $,
	
	\item {\color{deepgreen} \itshape важный признак} Число положительных $ n_i^{+} $ и отрицательных $ n_i^{-} $ коэффициентов в ограничениях, ассоциированных с рассматриваемой переменной $ x_i $,
	
	\item Булев маркер удаления переменной $ x_i $ после шага снижения размерности задачи,
	
	\item {\color{deepgreen} \itshape важный признак} Коэффицент $ c_i $ при переменной $ x_i $ в целевой функции $ \mathbf{c}^T \mathbf{x} $,
	
	\item {\color{red} \itshape бесполезный признак} Вероятность\footnote{Идея построения признака основана на способе вычисления вероятности единичного выхода нейрона в машинах Больцмана \cite[\strbook{653}]{geron:ml-2018}} того, что $ i $-ая бинарная или целочисленная переменная $ x_i $ будет выставлена в~1 (индекс <<$ {-i} $>> означает без учета $ i $-ой переменной)
	\begin{align*}
		\mathbf{P}(x_i = 1) = \sigma \bigg( \dfrac{1}{t} \, (\mathbf{c}^T \mathbf{x})_{-i} \bigg),
	\end{align*}
	где $ \sigma $ -- логистический сигмоид, $ t $ -- <<температура>> (чем выше температура, тем случайнее выход), $ \mathbf{c} $ -- вектор коэффициентов целевой функции, $ \mathbf{x} $ -- вектор значений переменных в релаксированном решении.
	
	\item Важность $ x_i $ переменной с точки зрения пресолверов.
\end{enumerate}

\subsection{Стратегии решения задачи}

\subsubsection{Стратегия №1. Обнаружение аномалий}\label{sec:obj_detect}

%С помощью техник t-SNE и LLE найти низкоразмерное представление бинарно-целочисленных переменных. Оценить пермутационную важность признаков и важность признаков по Шепли.
%
%Оценить меру похожести \emph{релаксированного решения} $ {}^r\{x_k\}_{k=1}^M $ и \emph{допустимого целочисленного решения} $ {}^f\{x_k\}_{k=1}^M $, например, с помощью \emph{коэффициента Отиаи}\footnote{\url{https://en.wikipedia.org/wiki/Cosine_similarity}}
%\begin{align*}
%	K = \bigg[ \dfrac{ \# {}^r\{x_k\}\bigcap {}^f\{x_k\} }{M} \bigg]^2, \quad K \in [0, 1],
%\end{align*}
%где $ {}^r\{x_k\} $ -- набор значений переменных в релаксированном решении; $ {}^f\{x_k\} $ -- набор значений переменных в допустимом решении; $ \# x \bigcap y $ -- количество переменных на пересечении решений $ x $ и $ y $; $ M $ -- количество переменных в сценарии.
%
%Если релаксированное решение и допустимое целочисленное не пересекаются, то есть не имеют переменных с одним и тем же значением, то очевидно коэффициент Отиаи равен нулю. Если же решения пересекаются по всем переменным, то коэффициент становится равным единице. 

Задачу построения частично-заданного решения на фиксациях предлагается свести к задаче обнаружения аномалий в данных. Бинарные и целочисленные переменные, которые {как ожидается} примут \emph{нулевые значения} в допустимом целочисленном решении будем считать <<\underline{штатным}>> \underline{режимом}, а бинарные и целочисленные переменные, которые {как ожидается} примут \emph{ненулевые значения} в допустимом целочисленном решении -- \underline{аномалиями}. Такие <<аномальные>> экзмепляры остаются без рекомендуемого значения для фиксации, а оставшиеся нулевые <<штатные>> бинарные и целочисленные переменные фиксируются в ноль и на этом процедура построения частично-заданного решения считается завершенной.

Для повышения надежности прогноза предлагается использовать ансамбль детекторов аномалий. Решение о фиксации бинарной или целочисленной переменной в ноль принимается на основании большинства голосов ансамбля детекторов.

Набор данных представляет собой неупорядоченную коллекцию матриц признакового описания, ассоциированных с соответствующими lp/mps-файлами математической постановки задачи (условимся называть их \emph{сценариями}).



Ансамбль детекторов аномалий обучается по роторной схеме:
\begin{itemize}
	\item На $ i $-ой итерации все \emph{матрицы признакового описания} (всего в наборе $ S $ матриц/сценариев) кроме $ i $-ой матрицы используются для обучения детекторов, а на $ i $-ой матрице признакового описания строится прогноз аномальных экземпляров, которые помечаются как <<-1>>. В результате получается коллекция бинарных и целочисленных переменных, помеченных либо как <<0>>, либо как <<-1>>. Построенное решение сравнивается с допустимым целочисленным решением с помощью различных метрик качества (параметрическое гармоническое среднее, каппа Коэна, коэффициент корреляции Метьюса и т.д.). Вычисленные для $ i $-ой матрицы метрики качества и построенное частично-заданное решение на фиксациях сохраняются в директории результатов,
	
	\item Затем описанный шаг повторяется для оставшихся матриц признакового описания объекта.
\end{itemize}

По окончании процедуры для каждого сценария:
\begin{itemize}
	\item будут вычислены метрики качества,
	
	\item будет построенно частично-заданное решение на фиксациях,
\end{itemize}

Полученные частично-заданные решения на фиксациях подаются на вход решателю SCIP. Если SCIP удалось найти решение, обозначаемое как $ s_{\text{ML}} $, то оно сравнивается с решением $ s_{\text{FZB}} $, полученным с помощью метаконфигурации FZBIVSUHPB (см. подраздел~\ref{sec:ikp_bins}), по времени работы и по значению верхней гранцы решения.

\remark{
Как правило, в задачах обнаружения аномалий не выполняют подбор гиперпараметров детектора, но в данном случае кажется полезным изучить поведение детектора хотя бы в зависимости от параметра контаминации. Дело в том, что на практике эффективность детектора может существенно изменяться в зависимости от значений управляющих параметров
}

На всех сценариях группы ИКП (см. раздел~\ref{sec:ikp}) обнаруживается серьезный дисбаланс экземпляров положительного (<<аномалии>>, ненулевые значения переменных) и отрицательного (<<штатные>> экземпляры, нулевые значения переменных) классов. Ожидается, что эффективность модели машинного обучения главным образом будет зависеть от способности модели выявлять аномальные экземплеры.

Действительно, \emph{ошибка первого рода} (ложное срабатывание, т.е. когда отрицательный <<штатный>> экземпляр принимается за <<аномальный>> положительный) приводит к тому, что нулевая переменная \emph{не будет} зафиксирована в ноль в частично-заданном решении, что с высокой вероятностью снизит производительность решателя SCIP.

Тогда как \emph{ошибка второго рода} (пропуск объекта, т.е. когда <<аномальный>> положительный экземпляр принимается за <<штатный>> отрицательный) приводит к тому, что ненулевая переменная в частично-заданном решении будет зафиксирована в ноль. Это сделает частично-заданное решение не способным развиться в допустимое целочисленное, что значительно хуже.

Таким образом, кажется разумным сосредоточить усилия на том, чтобы минимизировать {ошибку второго рода}, и в результате свести к минимуму число пропусков аномалий.

Проще всего оценить качество модели с учетом б\emph{о}льшего влияния ошибок второго рода с помощью \href{https://scikit-learn.org/stable/modules/model_evaluation.html#precision-recall-f-measure-metrics}{$ F_{\beta} $-меры} при значениях параметра $ \beta > 1 $
\begin{align*}
	F_{\beta} = (1 + \beta^2) \, \dfrac{\text{precision} \cdot \text{recall}}{\beta^2\, \text{precision} + \text{recall}},
\end{align*}
где $ \text{precision} $ -- точность, $ \text{recall} $ -- полнота.

\remark{
Провести анализ приема подбора порога бинаризации. И проработать схему подбора гиперпараметров детекторов
}

\paragraph{Анализ производительности методов обнаружения аномалий} 

Рекомендуемые значения некотрых гиперпараметров для детекторов некоторых семейств звучат следующим образом \cite{soenen:effect-hyper-param-tuning:2021}:
\begin{itemize}
	\item для KNN (k Nearest Neighbors\footnote{Расстояние от $ k $-ого ближайшего соседа рассматривается как мера аномальности экземпляра}) и LOF (Local Outlier Factor): $ k = \max (10; 0.03 \, |D|) $, где $ |D| $ - число экземпляров в наборе данных,
	
	\item для HBOS (Histogram-based Outlier Score): \texttt{n\_bins} $ = \sqrt{|D|} $,
	
	\item для IForest (Isolation Forest): число деревьев \texttt{n\_estimators=100 } и число экземпляров на дерево \texttt{max\_samples=256},
	
	\item для CBLOF (Clustering-Based Local Outlier Factor): $ \alpha = 0.90 $, $ \beta = 5 $ и $ k = 10 $,
	
	\item для OCSVM (One-Class Support Vector Machines): ядро RFB($ \nu = 0.5, \gamma = 1/m $), где $ m $ - число признаков в наборе данных $ D $.
\end{itemize}
\vspace*{2mm}

Перечисленные ниже детекторы показали крайне низкую производительность на сценариях группы ИКП: 
\begin{itemize}
	\item KNN,
	
	\item Feature Bagging,
	
	\item ABOD (Angle-Based Outlier Detection using approximation)/FastABOD,
	
	\item LOCI (Fast outlier detection using the local correlation integral),
	
	\item CBLOF (Clustering-Based Local Outlier Factor): достаточно быстрый, но результаты отвратительные (очень низкие значения ключевых метрик качества),
	
	\item XGBOOD\footnote{Требует разметки} (Extreme Boosting Based Outlier Detection): безумно медленный\footnote{В \url{https://github.com/yzhao062/pyod/issues/152} рекомендуется использовать SUOD},
	
	\item R-Graph (Outlier detection by R-graph).
\end{itemize}
\vspace*{2mm}

Главный детектор аномалий предлагается строить с помощью агрегатора SUOD\footnote{\url{https://www.andrew.cmu.edu/user/yuezhao2/papers/21-mlsys-suod.pdf}} (Accelerating Large-scale Unsupervised Heterogeneous Outlier Detection) на следующих базовых детекторах:
\begin{itemize}
	\item ECOD (Unsupervised Outlier Detection Using Empirical Cumulative Distribution Functions),
	
	\item COPOD (Copula-Based Outlier Detection),
	
	\item IForest (Isolation Forest),
	
	\item HBOS (Histogram-based Outlier Score).
\end{itemize}



\subsubsection{Стратегия №2. Бинарная классификация со слабо выраженным миноритарным классом} Задачу построения частично-заданного решения на фиксациях предлагатеся свести к задаче бинарной классификации со слабо выраженным миноритарным классом (данные с сильным дисбалансом).

\emph{Раздел в разработке ...}

\subsection{Трансфер выявленного паттерна. Сценарии группы СОП}

Условимся \emph{трансфером выявленного паттерна} (или просто \emph{трансфером паттерна}) называть являение, состоящее в том, что модель, обученная на сценариях одной группы (сценарии обучающего поднабора), оказывается способной строить корректные прогнозы на сценариях другой группы (сценарии тестового поднабора), обладающих четкими дискриминирующими атрибутами (структурные особенности матрицы ограничений и пр.), которые позволяют с высокой степенью уверенности отделять сценарии обучающего поднабора от сценариев тестового поднабора.

Другими словами, в отличие от классической постановки машинного обучения -- в которой экземпляры обучающего и тестового поднаборов данных должны быть похожи друг на друга -- в данном случае модель машинного обучения предлагается обучать и тестировать на сценариях, которые значимо отличаются друг от друга по каким-то ключевым аттрибутам.

\subsubsection{Сценарий \texttt{tmpfvpqodxw.lp} без бинарных переменных}

Исследование вопроса о трансфере паттерна начнем с рассмотрения простого сценария  группы СОП \texttt{tmpfvpqodxw.lp} \url{https://disk.yandex.ru/d/K7bvClpltotqlg}, а обучать модель машинного обучения будем в соответствие со стратегией №1 (\str{sec:obj_detect}).

В случае сценария \texttt{tmpfvpqodxw.lp} для простоты можно ограничиться рассмотрением только детектора HBOS (без агрегации прогнозов других детекторов с помощью обертки SUOD) и обучать его на сценарии группы ИКП \texttt{f398266b\_bin.lp} (см. раздел~\ref{sec:f398266b_bin}).

Для того чтобы использовать не ансамбль детекторов аномалий, а лишь какой-то конкретный детектор, достаточно в конфигурационном файле \texttt{main\_config.yaml} передать полю \texttt{use} детектора значение \texttt{False}
\begin{lstlisting}[
title = {\sffamily main\_config.yaml. Использовать только детектор HBOS},
style = bash,
numbers = none
]
...
detector_config:
	# Строит ансамбль детекторов аномалий
	SUOD:  # Scalable Unsupervised Outlier Detection https://www.andrew.cmu.edu/user/yuezhao2/papers/21-mlsys-suod.pdf
		use: !!bool False  # <--- NB
		# Допустимые значения 'combination': average, maximization
		combination: !!str average  # стратегия агрегации прогнозов ансамбля детектеров
		contamination: !!float 0.10  # доля выбросов в наборе данных; принимает значения из диапазона (0.0; 0.5)
		n_jobs: -1  # число параллельно выполняемых задач
		verbose: !!bool True  # флаг подробного вывода информации о построении модели
	# Перечень детекторов для SUOD-ансамбля. Если SUOD.use=True, то перечисленные ниже детекторы, у которых
	# атрибут DETECTOR.use=True, будут добавлены в список SUOD().base_estimators.
	# Если SUOD.use=False, то поиск аномалий будет выполняться с помощью одного из приведенных ниже детекторов,
	# у которого атрибут DETECTOR.use=True
	COPOD:  # Copula Based Outlier Detector
		use: !!bool False  # <--- NB
		contamination: !!float 0.10  # доля выбросов в наборе данных; принимает значения из диапазона (0.0; 0.5)
		n_jobs: -1  # число параллельно выполняемых задач
	ECOD:  # Unsupervised Outlier Detection Using Empirical Cumulative Distribution Functions
		use: !!bool False  # <--- NB
		contamination: !!float 0.10  # доля выбросов в наборе данных; принимает значения из диапазона (0.0; 0.5)
		n_jobs: -1  # число параллельно выполняемых задач
	IForest:  # Wrapper of scikit-learn Isolation Forest with more functionalities
		use: !!bool False  # <--- NB
		n_estimators: !!int 250  # число деревьев принятния решений в лесе
		contamination: !!float 0.10  # доля выбросов в наборе данных; принимает значения из диапазона (0.0; 0.5)
		n_jobs: -1  # число параллельно выполняемых задач
	HBOS:  # Histogram-based outlier detection
		use: !!bool True  # <--- NB
		n_bins: !!int 10  # число бинов для построения гистограммы
		alpha: !!float 0.05  # параметр регуляризации
		contamination: !!float 0.10  # доля выбросов в наборе данных; принимает значения из диапазона (0.0; 0.5)
\end{lstlisting}

Приведенный на рис.~\ref{fig:summary_tmpfvpqodxw} график показывает, что
\begin{itemize}
	\item настройки решателя SCIP, ответственные за выбор переменных при \emph{ветвлении}\footnote{Параметр \texttt{branching/preferbinary}} и \emph{разрешении конфликтов}\footnote{Параметр \texttt{conflict/preferbinar}}, а также прием подавления подгруппы первичных эвристик низкой эффективности помогают снизить временные издержки при незначительном ухудшении целевой функции (зеленая кривая) относительно решения, полученного с помощью решателя SCIP с настройками по умолчанию (красная кривая),
	
	\item дополнительное снижение временных затрат можно получить подбором гиперпараметров детектора\footnote{В данном случае подбирался только гиперпараметр контаменации} (синяя кривая).
\end{itemize}

\begin{figure}[!h]
	\centering
	\includegraphics[scale=0.7]{figures/tmpfvpqodxw_edo.pdf}
	\caption{ Сводка результатов вычислительных экспериментов \\на сценарии группы СОП \texttt{tmpfvpqodxw.lp} }\label{fig:summary_tmpfvpqodxw}
\end{figure}

Детектору аномалий HBOS с подбором параметра контаменации (\texttt{contamination=0.04})\footnote{В библиотеке PyOD все детекторы аномалий имеют контаминацию уровня 0.10} удалось снизить количество бинарных переменных -- на 98, ограничений -- на 177, а  временные издержки снизились в 2.38 раза.


\subsubsection{Синтетический сценарий \texttt{1664182546\_82382.lp} с бинарными переменными}

\textbf{Статистика}\footnote{В скобках указана размерность задачи после шага пресолвинга с фиксацией, полученной с помощью ансамбля детекторов аномалий без подбора гиперпараметров детекторов}\vspace*{1mm}

Общее количество переменных: 5100 (4123)

Количество целочисленных переменных: 0 (0)

Количество бинарных переменных: 1768 (1132)

Количество ограничений: 11193 (10461)

lp-файл: \url{https://disk.yandex.ru/d/FuEBWt4zvFIsEA}

Блок подавления подгруппы первичных эвристик низкой эффективности конфигурационного файла SCIP
\begin{lstlisting}[
title = {\sffamily Фрагмент файла scip.set. Сценарий 1664182546\_82382.lp},
style = bash,
numbers = none
]
...
heuristics/adaptivediving/freq = -1
heuristics/fracdiving/freq = -1
heuristics/linesearchdiving/freq = -1
heuristics/objpscostdiving/freq = -1
heuristics/pscostdiving/freq = -1
heuristics/rootsoldiving/freq = -1
heuristics/veclendiving/freq = -1
\end{lstlisting}

На рис.~\ref{fig:1664182546_82382} приведены результаты сравнительного анализа запусков i) решателя SCIP с настройками по умолчанию, ii) решателя SCIP на частично-заданном решении, полученном с помощью ансамбля детекторов аномалий, и iii) решателя SCIP на фиксации, подговтоленной с помощью изолированного детектора HBOS.

Как видно из рисунка, решатель SCIP с настройками по умолчанию (синяя кривая) первое допустимое целочисленное решение с адекватным зазором находит гораздо позже схемы на частично-заданном решении, полученном с помощью ансамбля детекторов (красная кривая). Однако, спустя некоторое время схема с настройками по умолчанию быстрее выходит на конкурентное значение целевой функции (41389.75 против 41557.30).

Схема с подбором гиперпараметра контаменации изолированного детектора HBOS, несмотря на то, что размерность задачи снижается, приводит к очень слабому решению. 

\textbf{Вывод по сценарию}: принимая во внимание, что ансамбль детекторов аномалий обучался лишь на одном сценарии группы ИКП, который существенно и с точки зрения размерности, и с точки зрения структуры матрицы ограничений отличается от сценария, на котором строился прогнгоз модели, допустимо говорить о трансфере/переносе шаблона, выявленного на сценарии \texttt{f398266b\_bin.lp} группы ИКП.

\begin{figure}[!h]
	\centering
	\includegraphics[scale=0.58]{figures/1664182546_82382.pdf}
	\caption{ Сводка результатов вычислительных экспериментов на сценарии \\группы СОП \texttt{1664182546\_82382.lp} }\label{fig:1664182546_82382}
\end{figure}


\subsubsection{Синтетический сценарий \texttt{1664182533\_1587787.lp} с бинарными переменными}

\textbf{Статистика}\footnote{В скобках указана размерность задачи после шага пресолвинга с фиксацией, полученной с помощью ансамбля детекторов аномалий без подбора гиперпараметров детекторов}\vspace*{1mm}

Общее количество переменных: 4759 (3780)

Количество целочисленных переменных: 0 (0)

Количество бинарных переменных: 1701 (1063)

Количество ограничений: 10307 (9581)

lp-файл: \url{https://disk.yandex.ru/d/n0Dqn6pr6GK9mg}

Блок подавления подгруппы первичных эвристик низкой эффективности конфигурационного файла SCIP
\begin{lstlisting}[
title = {\sffamily Фрагмент файла scip.set. Сценарий 1664182533\_1587787.lp},
style = bash,
numbers = none
]
...
heuristics/adaptivediving/freq = -1
heuristics/fracdiving/freq = -1
heuristics/linesearchdiving/freq = -1
heuristics/objpscostdiving/freq = -1
heuristics/pscostdiving/freq = -1
heuristics/rootsoldiving/freq = -1
heuristics/veclendiving/freq = -1
\end{lstlisting}

На рис.~\ref{fig:1664182533_1587787} приведены результаты сравнительного анализа запусков i) решателя SCIP с настройками по умолчанию, ii) решателя SCIP на частично-заданном решении, полученном с помощью ансамбля детекторов аномалий, и iii) решателя SCIP на фиксации, подговтоленной с помощью изолированного детектора HBOS.

Здесь схема с настройками по умолчанию проигрывает схеме на частично-заданном решении, построенном с помощью ансамбля детекторов аномалий, и по времени расчета, и по значению целевой функции. Подбор параметра контаминации детектора HBOS как и в предыдущем случае не позволяет улучшить решение -- кривая <<замирает>> на асимптоте 52070.46.

Таким образом, в данном случае ансамбль детекторов аномалий с обреткой SUOD снижает временные издержки на получение решения и одновременно улучшает целевую функцию.

\textbf{Вывод по сценарию}: принимая во внимание, что ансамбль детекторов аномалий обучался лишь на одном сценарии группы ИКП, который существенно и с точки зрения размерности, и с точки зрения структуры матрицы ограничений отличается от сценария, на котором строился прогнгоз модели, допустимо говорить о трансфере/переносе шаблона, выявленного на сценарии \texttt{f398266b\_bin.lp} группы ИКП.

\begin{figure}[!h]
	\centering
%	\begin{mdframed}[backgroundcolor=gray!25,linecolor=gray!25]
	    \includegraphics[scale=0.535]{figures/1664182533_1587787.pdf}
	    \caption{ Сводка результатов вычислительных экспериментов \\на сценарии группы СОП \texttt{1664182533\_1587787.lp} }\label{fig:1664182533_1587787}
%	\end{mdframed}
\end{figure}


\subsubsection{Синтетический сценарий \texttt{1664182480\_4326847.lp} с бинарными переменными}

\textbf{Статистика}\footnote{В скобках указана размерность задачи после шага пресолвинга с фиксацией, полученной с помощью ансамбля детекторов аномалий без подбора гиперпараметров детекторов}\vspace*{1mm}

Общее количество переменных: 7123 (6445)

Количество целочисленных переменных: 0 (0)

Количество бинарных переменных: 1548 (1324)

Количество ограничений: 17696 (16805)

lp-файл: \url{https://disk.yandex.ru/d/f_6GH9mzzxAGQg}

Блок подавления подгруппы первичных эвристик низкой эффективности конфигурационного файла SCIP
\begin{lstlisting}[
title = {\sffamily Фрагмент файла scip.set. Сценарий 1664182480\_4326847.lp},
style = bash,
numbers = none
]
...
heuristics/adaptivediving/freq = -1
heuristics/fracdiving/freq = -1
heuristics/linesearchdiving/freq = -1
heuristics/objpscostdiving/freq = -1
heuristics/pscostdiving/freq = -1
heuristics/rootsoldiving/freq = -1
heuristics/veclendiving/freq = -1
\end{lstlisting}

На рис.~\ref{fig:1664182480_4326847} приведены результаты сравнительного анализа запусков i) решателя SCIP с настройками по умолчанию и ii) решателя SCIP на фиксации, подговтоленной с помощью изолированного детектора HBOS.

На рассматриваемом сценарии получить решение с помощью ансамбля детекторов аномалий за отведенное для поиска время не удалось, однако, изолированный детектор HBOS с подобранным параметром контаминации смог выйти на значение целевой функции 53682.08. Это решение проигрывает решению, полученному с помощью SCIP базовой конфигурации (47245.97), но тем не менее указывает жизнеспособность концепции использования стратегии обнаружения аномалий для построения частично-заданного решения на фиксациях с подбором параметра контаминации детекторов.

\textbf{Вывод по сценарию}: принимая во внимание, что ансамбль детекторов аномалий обучался лишь на одном сценарии группы ИКП, который существенно и с точки зрения размерности, и с точки зрения структуры матрицы ограничений отличается от сценария, на котором строился прогнгоз модели, допустимо говорить о трансфере/переносе шаблона, выявленного на сценарии \texttt{f398266b\_bin.lp} группы ИКП.


\begin{figure}[!h]
	\centering
	\includegraphics[scale=0.58]{figures/1664182480\_4326847.pdf}
	\caption{ Сводка результатов вычислительных экспериментов \\на сценарии группы СОП \texttt{1664182480\_4326847.lp} }\label{fig:1664182480_4326847}
\end{figure}


\subsubsection{Синтетический сценарий \texttt{1664182523\_380519.lp} с бинарными переменными}

\textbf{Статистика}\footnote{В скобках указана размерность задачи после шага пресолвинга с фиксацией, полученной с помощью ансамбля детекторов аномалий без подбора гиперпараметров детекторов}\vspace*{1mm}

Общее количество переменных: 4578

Количество целочисленных переменных: 0

Количество бинарных переменных: 1331

Количество ограничений: 10722

lp-файл: \url{https://disk.yandex.ru/d/i-FhZ9LD8ToeXg}

Блок подавления подгруппы первичных эвристик низкой эффективности конфигурационного файла SCIP
\begin{lstlisting}[
title = {\sffamily Фрагмент файла scip.set. Сценарий 1664182523\_380519.lp},
style = bash,
numbers = none
]
...
heuristics/adaptivediving/freq = -1
heuristics/fracdiving/freq = -1
heuristics/linesearchdiving/freq = -1
heuristics/objpscostdiving/freq = -1
heuristics/pscostdiving/freq = -1
heuristics/rootsoldiving/freq = -1
heuristics/veclendiving/freq = -1
\end{lstlisting}

На рис.~\ref{fig:1664182523_380519} приведены результаты сравнительного анализа запусков i) решателя SCIP с настройками по умолчанию, и ii) решателя SCIP на частично-заданном решении, полученном с помощью ансамбля детекторов аномалий.

Здесь ансамбль детекторов аномалий выигрывает 2.78 минуты при целевой функции, значение которой практически не отличается от значения целевой функции в решении, полученном с помощью решателя SCIP базовой конфигурации.

\begin{figure}[!h]
	\centering
	\includegraphics[scale=0.58]{figures/1664182523\_380519.pdf}
	\caption{ Сводка результатов вычислительных экспериментов \\на сценарии группы СОП \texttt{1664182523\_380519.lp} }\label{fig:1664182523_380519}
\end{figure}

\textbf{Вывод по сценарию}: принимая во внимание, что ансамбль детекторов аномалий обучался лишь на одном сценарии группы ИКП, который существенно и с точки зрения размерности, и с точки зрения структуры матрицы ограничений отличается от сценария, на котором строился прогнгоз модели, допустимо говорить о трансфере/переносе шаблона, выявленного на сценарии \texttt{f398266b\_bin.lp} группы ИКП.




\section{Описание вычислительных экспериментов на сценариях группы ИКП}\label{sec:ikp}

На всех сценариях группы ИКП (как с бинарными переменными, так и без них) решения удавалось найти с помощью \emph{метаконфигурации} (см. раздел~\ref{sec:ikp_bins}), включающей прием подавления подгруппы первичных эвристик низкой эффективности и процедуру построения частично-заданного решения на фиксациях (для нулевых бинарных и целочисленных переменных). 

\subsection{Поиск решения на сценариях \emph{без} бинарных переменных. \\Метаконфигурации SUH,  FZBIVSUHPB и ансамбль детекторов аномалий}

Метаконфигурация\footnote{Под метаконфигурацией понимается совокупность конфигурации решателя и набора эвристических приемов} SUH (Suppress Useless Heuristics) процедуры поиска решения сводится к приему подавления подгруппы первичных эвристик низкой эффективности.

\remark{
Решение получено без доменно-ориентированных эвристик, <<теплого>> старта и подбора параметров решателя
}

Конфигурация решателя SCIP для всех сценариев группы ИКП (без бинарных переменных) имеет вид
\begin{lstlisting}[
	title = {\sffamily scip.set. Сценарии группы ИКП без бинарных переменных},
	style = bash,
	numbers = none,
	]
	# критерии останова и перезапуска
	limits/time = 7200
	limits/gap = 0.02  # решение останавливается при зазоре <= 2%
	
	# подавление подгруппы первичных эвристик низкой эффективности
	heuristics/farkasdiving/freq = -1
	heuristics/feaspump/freq = -1
	heuristics/randrounding/freq = -1
	heuristics/shiftandpropagate/freq = -1
	heuristics/shifting/freq = -1
\end{lstlisting}

Сводка результатов вычислительных экспериментов доступна по ссылке \url{https://docs.google.com/document/d/1V9fZLT9cXkbVQ5BvMCwzKrAiASZ2v4-01Z68jVBZUBU/edit?usp=sharing}.

\subsubsection{Сценарий \texttt{F398266B} без бинарных переменных}

\textbf{Статистика}\vspace*{1mm}

Общее количество переменных: 774901

Количество целочисленных переменных: 172449

Количество бинарных переменных: 0

Количество ограничений: 650263

lp-файл: \url{https://disk.yandex.ru/d/o_eAb9475u5ueg}

\vspace*{5mm}\textbf{Анализ решения}\vspace*{1mm}

Пул решений задачи был найден с помощью следующих первичных эвристик:
\begin{itemize}
	\item INTSHIFING,
	
	\item RENS.
\end{itemize}

Файл решения задачи (метаконфигурация SUH) доступен по ссылке \url{https://disk.yandex.ru/d/URRnZ8soTaJEgQ}

Файл статистической сводки (метаконфигурация SUH) доступен по ссылке \url{https://disk.yandex.ru/d/N2tfhj1N6RczzA}

Файл решения задачи (метаконфигурация FZBIVSUHPB) доступен по ссылке \url{https://disk.yandex.ru/d/-y7p5FyJyYirkw}

Файл статистической сводки (метаконфигурация FZBIVSUHPB) доступен по ссылке \url{https://disk.yandex.ru/d/1JaMC9aFjubDbA}

\vspace*{3mm}
\textbf{Вывод по сценарию}: описанная выше метаконфигурация SUH приводит к решению задачи, которое оказывается по отношению к результату на доменно-ориентированных эвристиках (\verb|USE_RECALCULATION_ON_FLOW=true|) для последнего решения из пула допустимых целочисленных решений (ОС Linux Centos 7) на 1.063\% лучше в смысле целевой функции и на 10.20\% -- в смысле временных издержек (\pic{fig:summary_f398266b}). 

Метаконфигурация FZBIVSUHPB (подробнее в разделе~\ref{sec:ikp_bins}) по отношению к тому же результату на доменно-ориентированных эвристиках дает решение задачи, которое на 1.155\% лучше в смысле целевой функции и на 65.27\% -- в смысле временных издержек (\tblref{tab:f398266b_wo_bins}).

Синим цветом обозначен выигрыш в процентах.

{
	%\rowcolors{2}{white}{lightgray!15}
	\begin{table}[!h]
		\centering
		\caption{Сводка результатов анализа эффективности \\метаконфигураций SUH и FZBIVSUHPB. Сценарий \texttt{f398266b} без бинарных переменных}
		\begin{tabular}{ p{2.5cm} p{3.3cm} p{3.4cm} }
			\emph{Способ} & \emph{Полное время расчета, мин} & \emph{Верхняя граница решения, $ \times 10^{10} $} \\
			\hline\hline\\[-3.5mm]
			{CBC+DOH} & 21.38 & $ 5.905048 $ \\
			\hline
			SCIP+SUH & 19.27 {\color{blue} $\ +9.87 $\%} & $ 5.842154 $ {\color{blue} $\ +1.065 $\%} \\
			\hline
			SCIP+FZB... & 9.43 {\color{blue} $\ +55.89 $\%} & $ 5.836815 $ {\color{blue} $\ +1.155 $\%} \\
		\end{tabular}\label{tab:f398266b_wo_bins}
	\end{table}
}

\begin{figure}[!h]
	\centering
	\includegraphics[scale=0.6]{figures/summary_f398266b.pdf}
	\caption{Сводка результатов анализа эффективности метаконфигурации SUH. \\Сценарий \texttt{f398266b} без бинарных переменных}\label{fig:summary_f398266b}
\end{figure}


\subsubsection{Сценарий \texttt{50197DF7} без бинарных переменных}

\textbf{Статистика}\vspace*{1mm}

Общее количество переменных: 718464

Количество целочисленных переменных: 159332

Количество бинарных переменных: 0

Количество ограничений: 595797

lp-файл: \url{https://disk.yandex.ru/d/KO_xj9dkgUdcog}

\vspace*{5mm}\textbf{Анализ решения}\vspace*{1mm}

Пул решений задачи был найден с помощью следующих первичных эвристик:
\begin{itemize}
	\item INTSHIFING,
	
	\item RENS.
\end{itemize}

Файл решения задачи (метаконфигурация SUH) доступен по ссылке \url{https://disk.yandex.ru/d/R4B1fkTx-nE3tg}

Файл статистической сводки (метаконфигурация SUH) доступен по ссылке \url{https://disk.yandex.ru/d/BLvUmZ43vtMFKg}

Файл решения задачи (метаконфигурация FZBIVSUHPB) доступен по ссылке \url{https://disk.yandex.ru/d/yMFLr-6mLfdPAw}

Файл статистической сводки (метаконфигурация FZBIVSUHPB) доступен по ссылке \url{https://disk.yandex.ru/d/XiRSvteL9xC4pg}

\vspace*{3mm}
\textbf{Вывод по сценарию}: описанная выше метаконфигурация SUH приводит к решению задачи, которое оказывается по отношению к результату на доменно-ориентированных эвристиках (\verb|USE_RECALCULATION_ON_FLOW=true|) для последнего решения из пула допустимых целочисленных решений (ОС Linux Centos 7) на 1.25\% лучше в смысле целевой функции и на 46.43\% -- в смысле временных издержек (\pic{fig:summary_50197df7}). 

Метаконфигурация FZBIVSUHPB (подробнее в разделе~\ref{sec:ikp_bins}) по отношению к тому же результату на доменно-ориентированных эвристиках дает решение задачи, которое на 1.191\% лучше в смысле целевой функции и на 82.13\% -- в смысле временных издержек (\tblref{tab:50197df7_wo_bins}).

Синим цветом обозначен выигрыш в процентах.

{
	%\rowcolors{2}{white}{lightgray!15}
	\begin{table}[!h]
		\centering
		\caption{Сводка результатов анализа эффективности \\метаконфигураций SUH и FZBIVSUHPB. Сценарий \texttt{50197df7} без бинарных переменных}
		\begin{tabular}{ p{2.5cm} p{3.3cm} p{3.4cm} }
			\emph{Способ} & \emph{Полное время расчета, мин} & \emph{Верхняя граница решения, $ \times 10^{10} $} \\
			\hline\hline\\[-3.5mm]
			{CBC+DOH} & 18.35 & $ 3.585532 $ \\
			\hline
			SCIP+SUH & 9.83 {\color{blue} $\ +46.43 $\%} & $ 3.540567 $ {\color{blue} $\ +1.252 $\%} \\
			\hline
			SCIP+FZB... & 3.28 {\color{blue} $\ +82.13 $\%} & $ 3.542843 $ {\color{blue} $\ +1.191 $\%} \\
		\end{tabular}\label{tab:50197df7_wo_bins}
	\end{table}
}

\begin{figure}[!h]
	\centering
	\includegraphics[scale=0.6]{figures/summary_50197df7.pdf}
	\caption{Сводка результатов анализа эффективности метаконфигурации SUH. \\Сценарий \texttt{50197df7} без бинарных переменных}\label{fig:summary_50197df7}
\end{figure}

\subsubsection{Сценарий \texttt{7FAC4231} без бинарных переменных}

\textbf{Статистика}\vspace*{1mm}

Общее количество переменных: 737585

Количество целочисленных переменных: 147789

Количество бинарных переменных: 0

Количество ограничений: 540018

lp-файл: \url{https://disk.yandex.ru/d/qiZAmraUNK1Peg}

\vspace*{5mm}\textbf{Анализ решения}\vspace*{1mm}

Пул решений задачи был найден с помощью следующих первичных эвристик:
\begin{itemize}
	\item INTSHIFING,
	
	\item RENS.
\end{itemize}

Файл решения задачи (метаконфигурация SUH) доступен по ссылке \url{https://disk.yandex.ru/d/20NeMuQ7NF_ccA}

Файл статистической сводки (метаконфигурация SUH) доступен по ссылке \url{https://disk.yandex.ru/d/QxE0HoREHzgHQQ}

Файл решения задачи (метаконфигурация FZBIVSUHPB) доступен по ссылке \url{https://disk.yandex.ru/d/FHZGj_Kyg8dDiw}

Файл статистической сводки (метаконфигурация FZBIVSUHPB) доступен по ссылке \url{https://disk.yandex.ru/d/8H1vw6zkQS7DAg}

\vspace*{3mm}
\textbf{Вывод по сценарию}: описанная выше метаконфигурация SUH приводит к решению задачи, которое оказывается по отношению к результату на доменно-ориентированных эвристиках (\verb|USE_RECALCULATION_ON_FLOW=true|) для последнего решения из пула допустимых целочисленных решений (ОС Linux Centos 7) на 5.22\% лучше в смысле целевой функции и на 27.10\% -- в смысле временных издержек (\pic{fig:summary_7fac4231}).

Метаконфигурация FZBIVSUHPB (подробнее в разделе~\ref{sec:ikp_bins}) по отношению к тому же результату на доменно-ориентированных эвристиках дает решение задачи, которое на 5.452\% лучше в смысле целевой функции и на 90.16\% -- в смысле временных издержек (\tblref{tab:7fac4231_wo_bins}).

Синим цветом обозначен выигрыш в процентах.

{
	%\rowcolors{2}{white}{lightgray!15}
	\begin{table}[!h]
		\centering
		\caption{Сводка результатов анализа эффективности \\метаконфигураций SUH и FZBIVSUHPB. Сценарий \texttt{7fac4231} без бинарных переменных}
		\begin{tabular}{ p{2.5cm} p{3.3cm} p{3.4cm} }
			\emph{Способ} & \emph{Полное время расчета, мин} & \emph{Верхняя граница решения, $ \times 10^{10} $} \\
			\hline\hline\\[-3.5mm]
			{CBC+DOH} & 16.05 & $ 1.087609 $ \\
			\hline
			SCIP+SUH & 11.67 {\color{blue} $\ +27.29 $\%} & $ 1.030866 $ {\color{blue} $\ +5.222 $\%} \\
			\hline
			SCIP+FZB... & 3.58 {\color{blue} $\ +77.69 $\%} & $ 1.028349 $ {\color{blue} $\ +5.452 $\%} \\
		\end{tabular}\label{tab:7fac4231_wo_bins}
	\end{table}
}

\begin{figure}[!h]
	\centering
	\includegraphics[scale=0.6]{figures/summary_7fac4231.pdf}
	\caption{Сводка результатов анализа эффективности метаконфигурации SUH. \\Сценарий \texttt{7fac4231} без бинарных переменных}\label{fig:summary_7fac4231}
\end{figure}

\subsubsection{Сценарий \texttt{CA485A55} без бинарных переменных}

\textbf{Статистика}\vspace*{1mm}

Общее количество переменных: 718601

Количество целочисленных переменных: 140858

Количество бинарных переменных: 0

Количество ограничений: 514229

lp-файл: \url{https://disk.yandex.ru/d/iSP6xrh4K_wHEQ}

\vspace*{5mm}\textbf{Анализ решения}\vspace*{1mm}

Пул решений задачи был найден с помощью следующих первичных эвристик:
\begin{itemize}
	\item INTSHIFING,
	
	\item RENS.
\end{itemize}

Файл решения задачи (метаконфигурация SUH) доступен по ссылке \url{https://disk.yandex.ru/d/_WzkmgoueNb2Bg}

Файл решения задачи (метаконфигурация FZBIVSUHPB) доступен по ссылке \url{https://disk.yandex.ru/d/sLUW5IxmpMBpcw}

Файл статистической сводки (метаконфигурация FZBIVSUHPB) доступен по ссылке \url{https://disk.yandex.ru/d/3Ls6QrAWVUMdZw}

\vspace*{3mm}
\textbf{Вывод по сценарию}: описанная выше метаконфигурация SUH приводит к решению задачи, которое оказывается по отношению к результату на доменно-ориентированных эвристиках (\verb|USE_RECALCULATION_ON_FLOW=true|) для последнего решения из пула допустимых целочисленных решений (ОС Linux Centos 7) на 0.683\% лучше в смысле целевой функции и на 46.48\% -- в смысле временных издержек (\pic{fig:summary_ca485a55}).

Метаконфигурация FZBIVSUHPB (подробнее в разделе~\ref{sec:ikp_bins}) по отношению к тому же результату на доменно-ориентированных эвристиках дает решение задачи, которое на 1.244\% лучше в смысле целевой функции и на 88.53\% -- в смысле временных издержек (\tblref{tab:ca485a55_wo_bins}).

Синим цветом обозначен выигрыш в процентах.

{
	%\rowcolors{2}{white}{lightgray!15}
	\begin{table}[!h]
		\centering
		\caption{Сводка результатов анализа эффективности \\метаконфигураций SUH и FZBIVSUHPB. Сценарий \texttt{ca485a55} без бинарных переменных}
		\begin{tabular}{ p{2.5cm} p{3.3cm} p{3.4cm} }
			\emph{Способ} & \emph{Полное время расчета, мин} & \emph{Верхняя граница решения, $ \times 10^{10} $} \\
			\hline\hline\\[-3.5mm]
			{CBC+DOH} & 20.05 & $ 4.597048 $ \\
			\hline
			SCIP+SUH & 10.73 {\color{blue} $\ +46.48 $\%} & $ 4.565579 $ {\color{blue} $\ +0.683 $\%} \\
			\hline
			SCIP+FZB... & 4.34 {\color{blue} $\ +78.35 $\%} & $ 4.539819 $ {\color{blue} $\ +1.244 $\%} \\
		\end{tabular}\label{tab:ca485a55_wo_bins}
	\end{table}
}

\begin{figure}[!h]
	\centering
	\includegraphics[scale=0.6]{figures/summary_ca485a55.pdf}
	\caption{Сводка результатов анализа эффективности метаконфигурации SUH. \\Сценарий \texttt{ca485a55} без бинарных переменных}\label{fig:summary_ca485a55}
\end{figure}

\subsubsection{Сценарий \texttt{276} без бинарных переменных}

\textbf{Статистика}\vspace*{1mm}

Общее количество переменных: 809224

Количество целочисленных переменных: 162562

Количество бинарных переменных: 0

Количество ограничений: 602190

lp-файл: \url{https://disk.yandex.ru/d/QaS5kd7VRZQ66A}

\vspace*{5mm}\textbf{Анализ решения}\vspace*{1mm}

Пул решений задачи был найден с помощью следующих первичных эвристик:
\begin{itemize}
	\item INTSHIFING,
	
	\item RENS.
\end{itemize}

Файл решения задачи (метаконфигурация SUH) доступен по ссылке \url{https://disk.yandex.ru/d/M2V88djiiGM5PA}

Файл решения задачи (метаконфигурация FZBIVSUHPB) доступен по ссылке \url{https://disk.yandex.ru/d/G0ustAVT6I9CeA}

Файл статистической сводки (метаконфигурация FZBIVSUHPB) доступен по ссылке \url{https://disk.yandex.ru/d/YBXB5GCECJiBIA}

\vspace*{3mm}
\textbf{Вывод по сценарию}: описанная выше метаконфигурация SUH приводит к решению задачи, которое оказывается по отношению к результату на доменно-ориентированных эвристиках (\verb|USE_RECALCULATION_ON_FLOW=true|) для последнего решения из пула допустимых целочисленных решений (ОС Linux Centos 7) на 3.67\% лучше в смысле целевой функции и на 51.56\% -- в смысле временных издержек (\pic{fig:summary_276}).

Метаконфигурация FZBIVSUHPB (подробнее в разделе~\ref{sec:ikp_bins}) по отношению к тому же результату на доменно-ориентированных эвристиках дает решение задачи, которое на 4.86\% лучше в смысле целевой функции и на 78.35\% -- в смысле временных издержек (\tblref{tab:276_wo_bins}).

Синим цветом обозначен выигрыш в процентах.

{
	%\rowcolors{2}{white}{lightgray!15}
	\begin{table}[!h]
		\centering
		\caption{Сводка результатов анализа эффективности \\метаконфигураций SUH и FZBIVSUHPB. Сценарий \texttt{276} без бинарных переменных}
		\begin{tabular}{ p{2.5cm} p{3.3cm} p{3.4cm} }
			\emph{Способ} & \emph{Полное время расчета, мин} & \emph{Верхняя граница решения, $ \times 10^{10} $} \\
			\hline\hline\\[-3.5mm]
			{CBC+DOH} & 29.87 & $ 1.430789 $ \\
			\hline
			SCIP+SUH & 14.47 {\color{blue} $\ +51.56 $\%} & $ 1.378299 $ {\color{blue} $\ +3.669 $\%} \\
			\hline
			SCIP+FZB... & 3.95 {\color{blue} $\ +78.35 $\%} & $ 1.361368 $ {\color{blue} $\ +4.857 $\%} \\
		\end{tabular}\label{tab:276_wo_bins}
	\end{table}
}


\begin{figure}[!h]
	\centering
	\includegraphics[scale=0.6]{figures/summary_276.pdf}
	\caption{Сводка результатов анализа эффективности метаконфигурации SUH. \\Сценарий \texttt{276} без бинарных переменных}\label{fig:summary_276}
\end{figure}

\subsubsection{Сценарий \texttt{337} без бинарных переменных}

\textbf{Статистика}\vspace*{1mm}

Общее количество переменных: 859075

Количество целочисленных переменных: 173622

Количество бинарных переменных: 0

Количество ограничений: 624327

lp-файл: \url{https://disk.yandex.ru/d/keyQLAagsD7Sbw}

\vspace*{5mm}\textbf{Анализ решения}\vspace*{1mm}

Пул решений задачи был найден с помощью следующих первичных эвристик:
\begin{itemize}
	\item INTSHIFING,
	
	\item RENS.
\end{itemize}

Файл решения задачи (метаконфигурация SUH) доступен по ссылке \url{https://disk.yandex.ru/d/ZUIEo3dDq77FjA}

Файл решения задачи (метаконфигурация FZBIVSUHPB) доступен по ссылке \url{https://disk.yandex.ru/d/0nUXIrIKuzqZlw}

Файл статистической сводки (метаконфигурация FZBIVSUHPB) доступен по ссылке \url{https://disk.yandex.ru/d/U0NCnMQN1akHUA}

\vspace*{3mm}
\textbf{Вывод по сценарию}: описанная выше метаконфигурация SUH приводит к решению задачи, которое оказывается по отношению к результату на доменно-ориентированных эвристиках (\verb|USE_RECALCULATION_ON_FLOW=true|) для последнего решения из пула допустимых целочисленных решений (ОС Linux Centos 7) на 22.12\% лучше в смысле целевой функции и на 18.32\% -- в смысле временных издержек (\pic{fig:summary_337}).

Метаконфигурация FZBIVSUHPB (подробнее в разделе~\ref{sec:ikp_bins}) по отношению к тому же результату на доменно-ориентированных эвристиках дает решение задачи, которое на 22.59\% лучше в смысле целевой функции и на 70.84\% -- в смысле временных издержек (\tblref{tab:337_wo_bins}).

Синим цветом обозначен выигрыш в процентах.

{
	%\rowcolors{2}{white}{lightgray!15}
	\begin{table}[!h]
		\centering
		\caption{Сводка результатов анализа эффективности \\метаконфигураций SUH и FZBIVSUHPB. Сценарий \texttt{337} без бинарных переменных}
		\begin{tabular}{ p{2.5cm} p{3.3cm} p{3.4cm} }
			\emph{Способ} & \emph{Полное время расчета, мин} & \emph{Верхняя граница решения, $ \times 10^{10} $} \\
			\hline\hline\\[-3.5mm]
			{CBC+DOH} & 20.85 & $ 3.825042 $ \\
			\hline
			SCIP+SUH & 17.03 {\color{blue} $\ +18.32 $\%} & $ 2.978782 $ {\color{blue} $\ +22.123 $\%} \\
			\hline
			SCIP+FZB... & 6.08 {\color{blue} $\ +70.84 $\%} & $ 2.961019 $ {\color{blue} $\ +22.588 $\%} \\
		\end{tabular}\label{tab:337_wo_bins}
	\end{table}
}


\begin{figure}[!h]
	\centering
	\includegraphics[scale=0.6]{figures/summary_337.pdf}
	\caption{Сводка результатов анализа эффективности метаконфигурации SUH. \\Сценарий \texttt{337} без бинарных переменных}\label{fig:summary_337}
\end{figure}

\subsubsection{Сценарий \texttt{13D686AB} без бинарных переменных}

\textbf{Статистика}\vspace*{1mm}

Общее количество переменных: 786020

Количество целочисленных переменных: 168857

Количество бинарных переменных: 0

Количество ограничений: 598414

lp-файл: \url{https://disk.yandex.ru/d/3KkYKzNl3PjGdg}

Пул решений задачи был найден с помощью следующих первичных эвристик:
\begin{itemize}
	\item INTSHIFING,
	
	\item RENS.
\end{itemize}

Файл решения задачи (метаконфигурация SUH) доступен по ссылке \url{https://disk.yandex.ru/d/EXylMeX6Ytz4tg}

Файл решения задачи (метаконфигурация FZBIVSUHPB) доступен по ссылке \url{https://disk.yandex.ru/d/dXUMVbSWRbqeDQ}

Файл статистической сводки (метаконфигурация FZBIVSUHPB) доступен по ссылке \url{https://disk.yandex.ru/d/Knavj89muxGw-w}

\vspace*{3mm}
\textbf{Вывод по сценарию}: описанная выше метаконфигурация SUH приводит к решению задачи, которое оказывается по отношению к результату на доменно-ориентированных эвристиках (\verb|USE_RECALCULATION_ON_FLOW=true|) для последнего решения из пула допустимых целочисленных решений (ОС Linux Centos 7) на 9.40\% лучше в смысле целевой функции и на 33.03\% -- в смысле временных издержек (\pic{fig:summary_13d686ab}).

Метаконфигурация FZBIVSUHPB (подробнее в разделе~\ref{sec:ikp_bins}) по отношению к тому же результату на доменно-ориентированных эвристиках дает решение задачи, которое на 10.44\% лучше в смысле целевой функции и на  75.82\% -- в смысле временных издержек (\tblref{tab:13d686ab_wo_bins}).

Синим цветом обозначен выигрыш в процентах.

{
	%\rowcolors{2}{white}{lightgray!15}
	\begin{table}[!h]
		\centering
		\caption{Сводка результатов анализа эффективности \\метаконфигураций SUH и FZBIVSUHPB. Сценарий \texttt{13d686ab} без бинарных переменных}
		\begin{tabular}{ p{2.5cm} p{3.3cm} p{3.4cm} }
			\emph{Способ} & \emph{Полное время расчета, мин} & \emph{Верхняя граница решения, $ \times 10^{9} $} \\
			\hline\hline\\[-3.5mm]
			{CBC+DOH} & 28.82 & $ 8.774743 $ \\
			\hline
			SCIP+SUH & 19.30 {\color{blue} $\ +33.03 $\%} & $ 7.949568 $ {\color{blue} $\ +9.403 $\%} \\
			\hline
			SCIP+FZB... & 6.97 {\color{blue} $\ +75.82 $\%} & $ 7.858548 $ {\color{blue} $\ +10.441 $\%} \\
		\end{tabular}\label{tab:13d686ab_wo_bins}
	\end{table}
}

\begin{figure}[!h]
	\centering
	\includegraphics[scale=0.6]{figures/summary_13d686ab.pdf}
	\caption{Сводка результатов анализа эффективности метаконфигурации SUH. \\Сценарий \texttt{13d686ab} без бинарных переменных}\label{fig:summary_13d686ab}
\end{figure}

\subsubsection{Сценарий \texttt{A78CBEAD} без бинарных переменных}

\textbf{Статистика}\vspace*{1mm}

Общее количество переменных: 795400

Количество целочисленных переменных: 180160

Количество бинарных переменных: 0

Количество ограничений: 658339

lp-файл: \url{https://disk.yandex.ru/d/vTPPa1H3VFD7tA}

Пул решений задачи был найден с помощью следующих первичных эвристик:
\begin{itemize}
	\item INTSHIFING,
	
	\item RENS.
\end{itemize}

Файл решения задачи (метаконфигурация SUH) доступен по ссылке \url{https://disk.yandex.ru/d/fARVcHb66ToHxQ}

Файл решения задачи (метаконфигурация FZBIVSUHPB) доступен по ссылке \url{https://disk.yandex.ru/d/4DItEZTja77cog}

Файл статистической сводки (метаконфигурация FZBIVSUHPB) доступен по ссылке \url{https://disk.yandex.ru/d/vn1K834mY5MEng}

\vspace*{3mm}
\textbf{Вывод по сценарию}: описанная выше метаконфигурация SUH приводит к решению задачи, которое оказывается по отношению к приему на доменно-ориентированных эвристиках (\verb|USE_RECALCULATION_ON_FLOW=true|) для последнего решения из пула допустимых целочисленных решений (ОС Linux Centos 7) на 1.57\% лучше в смысле целевой функции и на 23.30\% -- в смысле временных издержек (\pic{fig:summary_a78cbead}).

Метаконфигурация FZBIVSUHPB (подробнее в разделе~\ref{sec:ikp_bins}) по отношению к приему построения решения на доменно-ориентированных эвристиках дает решение задачи, которое на 1.39\% лучше в смысле целевой функции и на  81.04\% -- в смысле временных издержек (\tblref{tab:a78cbead_wo_bins}).

Синим цветом обозначен выигрыш в процентах.

{
	%\rowcolors{2}{white}{lightgray!15}
	\begin{table}[!h]
		\centering
		\caption{Сводка результатов анализа эффективности \\метаконфигураций SUH и FZBIVSUHPB. Сценарий \texttt{a78cbead} без бинарных переменных}
		\begin{tabular}{ p{2.5cm} p{3.3cm} p{3.4cm} }
			\emph{Способ} & \emph{Полное время расчета, мин} & \emph{Верхняя граница решения, $ \times 10^{10} $} \\
			\hline\hline\\[-3.5mm]
			{CBC+DOH} & 26.05& $ 3.801546 $ \\
			\hline
			SCIP+SUH & 19.98 {\color{blue} $\ +23.30 $\%} & $ 3.741685 $ {\color{blue} $\ +1.576 $\%} \\
			\hline
			SCIP+FZB... & 4.94 {\color{blue} $\ +81.04 $\%} & $ 3.748890 $ {\color{blue} $\ +1.386 $\%} \\
		\end{tabular}\label{tab:a78cbead_wo_bins}
	\end{table}
}

\begin{figure}[!h]
	\centering
	\includegraphics[scale=0.6]{figures/summary_a78cbead.pdf}
	\caption{Сводка результатов анализа эффективности метаконфигурации SUH. \\Сценарий \texttt{a78cbead} без бинарных переменных}\label{fig:summary_a78cbead}
\end{figure}

\subsubsection{Сценарий \texttt{496} (hard) без бинарных переменных}

\textbf{Статистика}\footnote{В первых скобках указана размерность задачи после шага пресолвинга с фиксацией FZBIVSUHPB, а во вторых -- с фиксацией, полученной с помощью ансамбля детекторов аномалий без подбора гиперпараметров детекторов}\vspace*{1mm}

Общее количество переменных: 864743 (48862) (90762)

Количество целочисленных переменных: 177365 (5008) (25872)

Количество бинарных переменных: 0 (332) (27)

Количество ограничений: 610819 (25438) (39119)

lp-файл: \url{https://disk.yandex.ru/d/CUA7wSn35k7Gbw}

Решение задачи было найдено с помощью первичной эвристики {INTSHIFTING}.

Файл решения задачи (метаконфигурация FZBIVSUHPB) доступен по ссылке \url{https://disk.yandex.ru/d/tbMiAbYmaAOrhg}

Файл статистической сводки (метаконфигурация FZBIVSUHPB) доступен по ссылке \url{https://disk.yandex.ru/d/AQptE3s3NF4bug}

Файл решения задачи (ансамбль детекторов аномалий) доступен по ссылке \url{https://disk.yandex.ru/d/VMZLFWoT8OftXA}

Файл статистической сводки (ансамбль детекторов аномалий) доступен по ссылке \url{https://disk.yandex.ru/d/KckqXgoKfv2fyQ}

Решение SCIP+ML получено с помощью ансамбля детекторов аномалий без подбора гиперпараметров детекторов.

\vspace*{3mm}
\textbf{Вывод по сценарию}: метаконфигурация FZBIVSUHPB (подробнее в разделе~\ref{sec:ikp_bins}) по отношению к приему на доменно-ориентированных эвристиках CBC+MS (measure of similarity) дает решение задачи, которое на 9.823\% лучше в смысле целевой функции и на  69.13\% -- в смысле временных издержек (\tblref{tab:496_wo_bins}).

Решение, полученное с помощью ансамбля детекторов аномалий, обученного на сценарии \texttt{f398266b\_bin.lp}, на 9.678\% превосходит CBC+MS в смысле целевой функции и на 71.82\% -- в смысле временных издержек.

SCIP+ML(0.10)f -- решение, полученное с помощью ансамбля детекторов аномалий без подбора параметра контаменации при первом запуске приложения, SCIP+ML(0.10)e -- то же самое, при запуске приложения в <<исследовательском режиме>> (матрица ограничений обучающего поднабора данных и релаксированные решения не вычисляются повторно).

Синим цветом обозначен выигрыш в процентах.

{
	%\rowcolors{2}{white}{lightgray!15}
	\begin{table}[!h]
		\centering
		\caption{Сводка результатов анализа эффективности \\метаконфигураций FZBIVSUHPB и ансамбля детекторов аномалий. \\Сценарий \texttt{496} без бинарных переменных}
		\begin{tabular}{ p{2.7cm} p{3.3cm} p{3.4cm} }
			\emph{Способ} & \emph{Полное время расчета, мин} & \emph{Верхняя граница решения, $ \times 10^{7} $} \\
			\hline\hline\\[-3.5mm]
			{CBC+MS}* & 5.00 & $ 6.536728 $ \\
			\hline
			Gurobi 9.12 & 5.22 {\color{red} $ \ - 0.04\% $} & 5.834197 {\color{blue} $ \ +10.747\% $}\\
			\hline
			SCIP 7.0.3d** & 15.42 {\color{red} $\ -66.15 $\%} & $ 10.66377 $ {\color{red} $\ -38.702 $\%} \\
			\hline
			SCIP+FZB... & 1.54 {\color{blue} $\ +69.13 $\%} & $ 5.894658 $ {\color{blue} $\ +9.823 $\%} \\
			\hline
			SCIP+ML(0.10)f & 4.56 {\color{blue} $\ +8.8 $\%} & 5.904120 {\color{blue} $\ +9.678 $\%} \\
			\hline
			SCIP+ML(0.10)e & 1.51 {\color{blue} $\ +69.76 $\%} & 5.904120 {\color{blue} $\ +9.678 $\%} 
		\end{tabular}\label{tab:496_wo_bins}
	\end{table}
\vspace*{-5mm}\hspace{30mm}\small{* -- опорное решение}
\hspace{15mm}\small{** -- решение было прервано}
}


\subsubsection{Сценарий \texttt{514} (hard) без бинарных переменных}

\textbf{Статистика}\footnote{В первых скобках указана размерность задачи после шага пресолвинга с фиксацией FZBIVSUHPB, а во вторых -- с фиксацией, полученной с помощью ансамбля детекторов аномалий без подбора гиперпараметров детекторов}\vspace*{1mm}

Общее количество переменных: 775879 (77367) (120764)

Количество целочисленных переменных: 145292 (5817) (32895)

Количество бинарных переменных: 0 (30) (14)

Количество ограничений: 541040 (45892) (61074)

lp-файл: \url{https://disk.yandex.ru/d/jQqSqBKb6iG-vw}

Пул решений задачи был найден с помощью следующих первичных эвристик:
\begin{itemize}
	\item INTSHIFTING,
	
	\item RENS.
\end{itemize}

Файл решения задачи (метаконфигурация FZBIVSUHPB) доступен по ссылке \url{https://disk.yandex.ru/d/lN2FdsqwEQcVTQ}

Файл статистической сводки (метаконфигурация FZBIVSUHPB) доступен по ссылке \url{https://disk.yandex.ru/d/iIdbACgh59EpVg}

Файл решения задачи (ансамбль детекторов аномалий) доступен по ссылке \url{https://disk.yandex.ru/d/5kRyOUsIOatHsQ}

Файл статистической сводки (ансамбль детекторов аномалий) доступен по ссылке \url{https://disk.yandex.ru/d/rNUU8HmeBGLFRQ}

Решение SCIP+ML получено с помощью ансамбля детекторов аномалий без подбора гиперпараметров детекторов.

\vspace*{3mm}
\textbf{Вывод по сценарию}: метаконфигурация FZBIVSUHPB (подробнее в разделе~\ref{sec:ikp_bins}) по отношению к приему построения решения с помощью меры подобия CBC+MS (measure of similarity) дает решение задачи, которое на 18.616\% лучше в смысле целевой функции и на  51.82\% хуже в смысле временных издержек (\tblref{tab:514_wo_bins}).

Решение, полученное с помощью ансамбля детекторов аномалий\footnote{Решение принудительно останавливалось на 350 секунде (параметр \texttt{limits/softtime = 350})}, обученного на сценарии \texttt{f398266b\_bin.lp}, на 19.562\% превосходит CBC+MS в смысле целевой функции и на 6.31\% -- в смысле временных издержек.

SCIP+ML(0.10)f -- решение, полученное с помощью ансамбля детекторов аномалий без подбора параметра контаменации при первом запуске приложения, SCIP+ML(0.10)e -- то же самое, при запуске приложения в <<исследовательском режиме>> (матрица ограничений обучающего поднабора данных и релаксированные решения не вычисляются повторно).

Синим цветом обозначен выигрыш в процентах, а красным -- проигрыш.

{
	%\rowcolors{2}{white}{lightgray!15}
	\begin{table}[!h]
		\centering
		\caption{Сводка результатов анализа эффективности \\метаконфигураций FZBIVSUHPB и ансамбля детекторов аномалий. \\Сценарий \texttt{514} без бинарных переменных}
		\begin{tabular}{ p{2.7cm} p{3.3cm} p{3.4cm} }
			\emph{Способ} & \emph{Полное время расчета, мин} & \emph{Верхняя граница решения, $ \times 10^{9} $} \\
			\hline\hline\\[-3.5mm]
			{CBC+MS}* & 13.00 & $ 5.243829 $ \\
			\hline
			Gurobi 9.12 & 11.(6) {\color{blue} $ +10.31 $} & $ 4.239092 $ {\color{blue} $\ +19.160 $\%} \\
			\hline
			SCIP 7.0.3d** & 60.32 {\color{red} $\ -79.47 $\%} & $ 47.82659 $ {\color{red} $\ -89.036 $\%} \\
			\hline
			SCIP+FZB... & 26.98 {\color{red} $\ -51.82 $\%} & $ 4.267692 $ {\color{blue} $\ +18.616 $\%} \\
			\hline
			SCIP+ML(0.10)f & 12.171 {\color{blue} $\ +6.38 $\%} & 4.217134 {\color{blue} $\ +19.580 $\%} \\
			\hline
			SCIP+ML(0.10)e & 6.53  {\color{blue} $\ +49.77 $\%} & 4.217134 {\color{blue} $\ +19.580 $\%}
		\end{tabular}\label{tab:514_wo_bins}
	\end{table}
\vspace{-5mm}\hspace{30mm}\small{* -- опорное решение}
\hspace{15mm}\small{** -- решение было прервано}
}

\subsubsection{Сценарий \texttt{519} (hard) без бинарных переменных}

\textbf{Статистика}\footnote{В скобках указана размерность задачи после шага пресолвинга с фиксацией FZBIVSUHPB}\vspace*{1mm}

Общее количество переменных: 684412 (75034)

Количество целочисленных переменных: 159200 (5424)

Количество бинарных переменных: 0 (44)

Количество ограничений: 447182 (44735)

lp-файл: \url{https://disk.yandex.ru/d/MMvnnYXK4J4Xxw}

Пул решений задачи был найден с помощью следующих первичных эвристик:
\begin{itemize}
	\item INTSHIFING,
	
	\item ONEOPT,
	
	\item VECLENDI,
	
	\item LINESEARCH,
	
	\item RENS.
\end{itemize}

%Файл решения задачи (метаконфигурация FZBIVSUHPB) доступен по ссылке \url{https://disk.yandex.ru/d/lN2FdsqwEQcVTQ}

%Файл статистической сводки (метаконфигурация FZBIVSUHPB) доступен по ссылке \url{https://disk.yandex.ru/d/iIdbACgh59EpVg}

Файл решения задачи (метаконфигурация FZBIVSUHPB) доступен по ссылке \url{https://disk.yandex.ru/d/25B3mUiRYdid3A}

Файл статистической сводки (метаконфигурация FZBIVSUHPB) доступен по ссылке \url{https://disk.yandex.ru/d/L3TyaXp56rZjCA}

Файл решения задачи (ансамбль детекторов аномалий) доступен по ссылке \url{}

Файл статистической сводки (ансамбль детекторов аномалий) доступен по ссылке \url{}

Решение SCIP+ML получено с помощью ансамбля детекторов аномалий без подбора гиперпараметров детекторов.

\vspace*{3mm}
\textbf{Вывод по сценарию}: метаконфигурация FZBIVSUHPB (подробнее в разделе~\ref{sec:ikp_bins}) по отношению к приему построения решения на доменно-ориентированных эвристиках CBC+MS (measure of similarity) дает решение задачи, которое на \% лучше в смысле целевой функции и на  \% хуже в смысле временных издержек (\tblref{tab:519_wo_bins}).

Решение, полученное с помощью отдельного детектора аномалий, обученного на сценарии \texttt{f398266b\_bin.lp}, на \% превосходит CBC+MS в смысле целевой функции и на \% -- в смысле временных издержек.

SCIP+ML(0.10)f -- решение, полученное с помощью ансамбля детекторов аномалий без подбора параметра контаменации при первом запуске приложения, SCIP+ML(0.10)e -- то же самое, при запуске приложения в <<исследовательском режиме>> (матрица ограничений обучающего поднабора данных и релаксированные решения не вычисляются повторно).

Синим цветом обозначен выигрыш в процентах, а красным -- проигрыш.

{
	%\rowcolors{2}{white}{lightgray!15}
	\begin{table}[!h]
		\centering
		\caption{Сводка результатов анализа эффективности \\метаконфигураций FZBIVSUHPB и ансамбля детекторов аномалий. \\Сценарий \texttt{519} без бинарных переменных}
		\begin{tabular}{ p{3.3cm} p{3.3cm} p{3.4cm} }
			\emph{Способ} & \emph{Полное время расчета, мин} & \emph{Верхняя граница решения, $ \times 10^{7} $} \\
			\hline\hline\\[-3.5mm]
			{CBC+MS}* & 6.00 & $ 7.719212 $ \\
			\hline
			Gurobi 9.12 & 3.48 {\color{blue} $\ +42.00 $\%} & $ 7.062839 $ {\color{blue} $\ +8.503 $\%} \\
			\hline
			SCIP 7.0.3d** & 41.92 {\color{red} $\ -91.70 $\%} & $ 31.59748 $ {\color{red} $\ +77.647 $\%} \\
			\hline
			SCIP+FZB (a) & 5.23 {\color{blue} $\ +12.83 $\%} & $ 7.901148 $ {\color{red} $\ -2.302 $\%} \\
			\hline
			SCIP+FZB (b) & 28.83 {\color{red} $\ -79.19 $\%} & $ 7.374810 $ {\color{blue} $\ +4.462 $\%} \\
			\hline
			SCIP+ML(0.10)f & 42.07 {\color{red} $\ -85.74 $ \%} & 7.014369 {\color{blue} $\ +9.130 $\%} \\
			\hline
%			SCIP+ML(0.10)e & 34.08 {\color{red} $\ -82.39 $ \%} & 7.011104 {\color{blue} $\ +9.173 $\%} \\
%			\hline
%			SCIP+ML(0.0468)f & 29.97 {\color{red} $\ -79.98 $ \%} & 7.147467 {\color{blue} $\ +7.408 $\%} \\
%			\hline
%			SCIP+ML(0.0468)e & 21.95 {\color{red} $\ -72.67 $ \%} & 7.147467 {\color{blue} $\ +7.408 $\%} \\
		\end{tabular}\label{tab:519_wo_bins}
	\end{table}
\vspace{-5mm}\hspace{30mm}\small{* -- опорное решение}
\hspace{15mm}\small{** -- решение было прервано}
}


\subsection{Поиск решения на сценариях \emph{с} бинарными переменными. \\Метаконфигурация FZBIVSUHPB}\label{sec:ikp_bins}

На ранних стадиях изучения проблемы высокоразмерных сценариев с бинарными переменными, поиск решения осуществлялся в семь шагов:
\begin{enumerate}
	\item  Подавить подгруппу первичных эвристик низкой эффективности (см. раздел ~\ref{sec:suh}),
	
	\item При разрешении конфликтов и ветвлении\footnote{К сожалению, на сценариях группы ИКП с бинарными переменными решателю SCIP не удается найти решение в корне дерева} отдавать предпочтение бинарным переменным,
	
	\item Найти релаксированное решение задачи,
	
	\item Подобрать порог бинаризации на релаксированном решении для бинарных переменных (см. раздел~\ref{sec:find_bin_thresh}),
	
	\item Зафиксировать \emph{нулевые} 0-bin и \emph{единичные} 1-bin \emph{бинарные переменные}; подать фиксацию решателю,
	
	\item В решении, найденном на предыдущей итерации, зафиксировать \emph{нулевые целочисленные} 0-int и \emph{единичные бинарные} 1-bin \emph{переменные}; полученную фиксацию подать на вход решателю,
	
	\item В решении, полученном на предыдущей итерации, зафиксировать \emph{нулевые бинарные} 0-bin и \emph{целочисленные} 0-int  \emph{переменные}; фиксацию подать на вход решателю.
\end{enumerate}

Процедура поиска оказалась чувствительной к параметру \texttt{autorestartnodes}. Графическая интерпретация результатов вычислительных экспериментов с разверткой процедуры поиска верхней границы решения во времени приведена на рис.~\ref{fig:a78cbead_autorestartnodes_1_2_phase}, \ref{fig:a78cbead_autorestartnodes_3_phase}, \ref{fig:50197df7_autorestartnodes} и \ref{fig:7fac4231_autorestartnodes}.

Позже описанную процедуру удалось упростить и свести к следующей \emph{метаконфигурации} {FZBIVSUHPB} (Fixed Zero Binary and Integer Variables, Suppress Useless Heuristics, Prefer Binary):
\begin{enumerate}
	\item Подавить подгруппу первичных эвристик низкой эффективности,
	
	\item При разрешении конфликтов и ветвлении отдавать предпочтение \emph{бинарным} переменным,
	
	\item Зафиксировать \emph{нулевые бинарные} 0-bin и \emph{нулевые целочисленные} 0-int  \emph{переменные} в релаксированном решении (см. раздел~\ref{sec:bin_int_relax_fix}).
\end{enumerate}

Конфигурация решателя SCIP для всех сценариев группы ИКП (с бинарными переменными) имеет вид
\begin{lstlisting}[
	title = {\sffamily scip.set. Сценарии группы ИКП с бинарными переменными},
	style = bash,
	numbers = none,
	]
# критерии останова и перезапуска
limits/time = 7200
limits/autorestartnodes = -1
limits/gap = 0.02  # решение останавливается при зазоре <= 2%

# управление стратегиями анализа конфликтов и ветвления
conflict/preferbinary = True
branching/preferbinary = True

# подавление подгруппы первичных эвристик низкой эффективности
heuristics/farkasdiving/freq = -1
heuristics/feaspump/freq = -1
heuristics/randrounding/freq = -1
heuristics/shiftandpropagate/freq = -1
heuristics/shifting/freq = -1
\end{lstlisting}

Все эксперименты проводились на виртуальной машине Linux (Centos 7) Intel Core™ i7 (8~CPUs), 3.6GHz, RAM 16Gb.

Сводка результатов вычислительных экспериментов доступна по ссылке \url{https://docs.google.com/document/d/1V9fZLT9cXkbVQ5BvMCwzKrAiASZ2v4-01Z68jVBZUBU/edit?usp=sharing}.

Кодовая база решения доступна по ссылке \url{https://gitdp.zyfra.com/ds_and_math_users/ml-dl-in-operations-reaseearches.git}

\subsubsection{Сценарий \texttt{A78CBEAD} с бинарными переменными}

\textbf{Статистика}\vspace*{1mm}

Общее количество переменных: 797818

Количество целочисленных переменных: 180160

Количество бинарных переменных: 2418

Количество ограничений: 663175

lp-файл: \url{https://disk.yandex.ru/d/JbT3KR5Yi1ZomQ}

\vspace*{5mm}\textbf{Анализ решения}\vspace*{1mm}

Пул решений задачи был найден с помощью следующих первичных эвристик:
\begin{itemize}
	\item DISTRIBUTIOINDIVING,
	
	\item ONEOPT,
	
	\item GINS.
\end{itemize}

Фргамент лога сессии SCIP
\begin{lstlisting}[
style = bash,
numbers = none
]
...
 time | node  | left  |LP iter|LP it/n|mem/heur|mdpt |vars |cons |rows |cuts |sepa|confs|strbr|  dualbound   | primalbound  |  gap   | compl. 
d1790s|  1881 |  1668 |  1010k| 296.9 |distribu|  93 |  50k|  43k|  43k|   0 |  1 | 385 |3585 | 3.757279e+10 | 3.894342e+10 |   3.65%|   7.70%
d1790s|  1881 |  1668 |  1010k| 296.9 |distribu|  93 |  50k|  43k|  43k|   0 |  1 | 385 |3585 | 3.757279e+10 | 3.894341e+10 |   3.65%|   7.70%
i1792s|  1882 |  1667 |  1011k| 297.0 |  oneopt|  93 |  50k|  43k|  43k|8612 |  0 | 385 |3585 | 3.757279e+10 | 3.893993e+10 |   3.64%|   7.70%
1796s|  1900 |  1687 |  1016k| 297.0 |  3669M |  93 |  50k|  43k|  43k|8644 |  1 | 387 |3585 | 3.757279e+10 | 3.893993e+10 |   3.64%|   2.82%
L1902s|  1982 |  1769 |  1090k| 313.4 |    gins|  93 |  50k|  43k|  43k|8935 |  1 | 398 |3590 | 3.757279e+10 | 3.875897e+10 |   3.16%|   2.83%
L1912s|  1982 |  1769 |  1090k| 313.4 |    gins|  93 |  50k|  43k|  43k|8935 |  1 | 398 |3590 | 3.757279e+10 | 3.864257e+10 |   2.85%|   2.83%
i1920s|  1982 |  1769 |  1099k| 316.2 |  oneopt|  93 |  50k|  43k|  43k|8935 |  1 | 398 |3590 | 3.757279e+10 | 3.864241e+10 |   2.85%|   2.83%
1954s|  2000 |  1787 |  1133k| 325.5 |  3731M |  93 |  50k|  43k|  43k|9004 |  1 | 398 |3591 | 3.757279e+10 | 3.864241e+10 |   2.85%|   2.83%
\end{lstlisting}

Файл решения задачи доступен по ссылке \url{https://disk.yandex.ru/d/6FPE-S5VupA6iw}

Файл статистической сводки доступен по ссылке \url{https://disk.yandex.ru/d/9G-v54ywEK1TJA}

\vspace*{3mm}
\textbf{Вывод по сценарию}: описанная выше метаконфигурация приводит к решению задачи, которое оказывается по отношению к результату на доменно-ориентированных эвристиках для последнего решения из пула допустимых целочисленных решений на 2.46\% лучше в смысле целевой функции и на 19.64\% -- в смысле временных издержек (\tblref{tab:a78cbead}).

В \tblref{tab:a78cbead}  через SCIP+MC~$ (a) $ обозначается решение, построенное на метаконфигурации SCIP, отвечающее \emph{первому} допустимому целочисленному решению, верхняя граница которого не превышает верхнюю границу решения на доменно-ориентированных эвристиках, а через SCIP+MC~$ (b) $ -- решение, отвечающее \emph{последнему} допустимому целочисленному решению в наборе полученных.

Синим цветом обозначен выигрыш в процентах.

{
%\rowcolors{2}{white}{lightgray!15}
\begin{table}[!h]
	\centering
	\caption{Сводка результатов анализа эффективности\\метаконфигурации FZBIVSUHPB. Сценарий \texttt{a78cbead} с бинарными переменными}
	\begin{tabular}{ p{2.5cm} p{3.3cm} p{3.4cm} }
		\emph{Способ} & \emph{Полное время расчета, мин} & \emph{Верхняя граница решения, $ \times 10^{10} $} \\
		\hline\hline\\[-3.5mm]
		{CBC+DOH} & 39.82 & $ 3.961502 $ \\
		\hline
		SCIP+MC ($ a $) & 29.83 {\color{blue} $\ +25.09 $\%} & $ 3.894342 $ {\color{blue} $\ +1.70 $\%} \\
		\hline
		SCIP+MC ($ b $) & 32.00 {\color{blue} $\ +19.64 $\%} & $ 3.864241 $ {\color{blue} $\ +2.46 $\%} \\
	\end{tabular}\label{tab:a78cbead}
\end{table}
}

\subsubsection{Сценарий \texttt{7FAC4231} с бинарными переменными}

\textbf{Статистика}\vspace*{1mm}

Общее количество переменных: 740251

Количество целочисленных переменных: 147789

Количество бинарных переменных: 2666

Количество ограничений: 545350

lp-файл: \url{https://disk.yandex.ru/d/3NbbjfLW5zhejQ}

\vspace*{5mm}\textbf{Анализ решения}\vspace*{1mm}

Пул решений задачи был найден с помощью следующих первичных эвристик:
\begin{itemize}
	\item INTSHIFTING,
	
	\item ONEOPT,
	
	\item GINS,
	
	\item CROSSOVER,
	
	\item ALNS.
\end{itemize}

Фрагмент лога сессии SCIP
\begin{lstlisting}[
style = bash,
numbers = none
]
...
 time | node  | left  |LP iter|LP it/n|mem/heur|mdpt |vars |cons |rows |cuts |sepa|confs|strbr|  dualbound   | primalbound  |  gap   | compl. 
r 454s|   372 |   341 | 91171 | 102.3 |intshift| 309 |  41k|  33k|  34k|2788 |  5 |  57 |3711 | 1.053077e+10 | 1.309195e+10 |  24.32%|   0.78%
i 454s|   373 |   340 | 91171 | 102.0 |  oneopt| 309 |  41k|  33k|  34k|2788 |  0 |  57 |3711 | 1.053077e+10 | 1.308634e+10 |  24.27%|   0.78%
463s|   400 |   369 | 93623 | 101.3 |  2493M | 309 |  41k|  33k|  34k|2950 |  1 |  57 |3761 | 1.053077e+10 | 1.308634e+10 |  24.27%|   0.29%
L 507s|   473 |   442 |106991 | 113.9 |    gins| 309 |  41k|  33k|  34k|3084 |  1 |  57 |3813 | 1.053077e+10 | 1.297515e+10 |  23.21%|   0.29%
L 512s|   473 |   442 |106991 | 113.9 |    gins| 309 |  41k|  33k|  34k|3084 |  1 |  57 |3813 | 1.053077e+10 | 1.292548e+10 |  22.74%|   0.29%
L 522s|   473 |   442 |106991 | 113.9 |    gins| 309 |  41k|  33k|  34k|3084 |  1 |  57 |3813 | 1.053077e+10 | 1.289283e+10 |  22.43%|   0.29%
L 525s|   473 |   442 |106991 | 113.9 |    gins| 309 |  41k|  33k|  34k|3084 |  1 |  57 |3813 | 1.053077e+10 | 1.286340e+10 |  22.15%|   0.29%
i 529s|   473 |   442 |112279 | 125.1 |  oneopt| 309 |  41k|  33k|  34k|3084 |  1 |  57 |3813 | 1.053077e+10 | 1.285668e+10 |  22.09%|   0.29%
r 531s|   474 |   443 |120630 | 142.5 |intshift| 309 |  41k|  33k|  34k|3084 |  1 |  58 |3813 | 1.053077e+10 | 1.197786e+10 |  13.74%|   0.29%
i 532s|   474 |   373 |124926 | 151.6 |  oneopt| 309 |  41k|  33k|  34k|3084 |  1 |  58 |3813 | 1.053077e+10 | 1.197230e+10 |  13.69%|   0.29%
536s|   500 |   399 |126496 | 146.9 |  2579M | 309 |  41k|  33k|  34k|3181 |  1 |  58 |3822 | 1.053077e+10 | 1.197230e+10 |  13.69%|   0.29%
567s|   600 |   499 |158520 | 175.8 |  2613M | 309 |  41k|  33k|  34k|3641 |  1 |  60 |3933 | 1.053095e+10 | 1.197230e+10 |  13.69%|   0.29%
L 739s|   659 |   554 |189783 | 207.6 |    gins| 309 |  41k|  33k|  34k|4060 |  1 |  62 |3978 | 1.053095e+10 | 1.191898e+10 |  13.18%|   0.29%
i 741s|   660 |   555 |198453 | 220.4 |  oneopt| 309 |  41k|  33k|  34k|4060 |  1 |  62 |3981 | 1.053095e+10 | 1.191889e+10 |  13.18%|   0.30%
794s|   700 |   595 |236166 | 261.7 |  2689M | 309 |  41k|  33k|  34k|4418 |  1 |  62 |4010 | 1.053095e+10 | 1.191889e+10 |  13.18%|   0.32%
836s|   800 |   695 |277232 | 280.4 |  2728M | 309 |  41k|  33k|  34k|4757 |  1 |  64 |4027 | 1.053219e+10 | 1.191889e+10 |  13.17%|   0.32%
L 967s|   860 |   693 |295017 | 281.5 |crossove| 309 |  41k|  33k|  34k|5000 |  1 |  64 |4059 | 1.053219e+10 | 1.154287e+10 |   9.60%|   0.32%
i 968s|   860 |   693 |300734 | 288.1 |  oneopt| 309 |  41k|  33k|  34k|5000 |  1 |  64 |4059 | 1.053219e+10 | 1.154284e+10 |   9.60%|   0.32%
990s|   900 |   733 |312921 | 288.9 |  2793M | 309 |  41k|  33k|  34k|5288 |  1 |  64 |4139 | 1.053219e+10 | 1.154284e+10 |   9.60%|   0.33%
1042s|  1000 |   823 |346085 | 293.2 |  2816M | 309 |  41k|  33k|  34k|5725 |  1 |  65 |4281 | 1.053219e+10 | 1.154284e+10 |   9.60%|   0.33%
L1083s|  1003 |   826 |347173 | 293.4 |    alns| 309 |  41k|  33k|  34k|5747 |  2 |  65 |4284 | 1.053219e+10 | 1.153273e+10 |   9.50%|   0.33%
i1084s|  1004 |   827 |352908 | 298.8 |  oneopt| 309 |  41k|  33k|  34k|5747 |  1 |  65 |4284 | 1.053219e+10 | 1.118743e+10 |   6.22%|   0.33%
1113s|  1100 |   699 |373504 | 291.4 |  2860M | 309 |  41k|  33k|  34k|6055 |  3 |  65 |4323 | 1.053219e+10 | 1.118743e+10 |   6.22%|   0.44%
1140s|     1 |     0 |419115 |     - |  3039M |   0 |  41k|  34k|  34k|   0 |  0 |  65 |4323 | 1.053219e+10 | 1.118743e+10 |   6.22%| unknown
\end{lstlisting}

Файл решения задачи доступен по ссылке \url{https://disk.yandex.ru/d/TmA6hqFV87eGTg}

Файл статистической сводки доступен по ссылке \url{https://disk.yandex.ru/d/CsGV_oal4OTxOQ}

\vspace*{3mm}
\textbf{Вывод по сценарию}: описанная выше метаконфигурация приводит к решению задачи, которое оказывается по отношению к результату на доменно-ориентированных эвристиках для последнего решения из пула допустимых целочисленных решений на 3.38\% лучше в смысле целевой функции и на 33.07\% -- в смысле временных издержек (\tblref{tab:7fac4231}).

В \tblref{tab:7fac4231}  через SCIP+MC~$ (a) $ обозначается решение, построенное на метаконфигурации SCIP, отвечающее \emph{первому} допустимому целочисленному решению, верхняя граница которого не превышает верхнюю границу решения на доменно-ориентированных эвристиках, а через SCIP+MC~$ (b) $ -- решение, отвечающее \emph{последнему} допустимому целочисленному решению в наборе полученных.

Синим цветом обозначен выигрыш в процентах.

{
	%\rowcolors{2}{white}{lightgray!15}
	\begin{table}[!h]
		\centering
		\caption{Сводка результатов анализа эффективности\\метаконфигурации FZBIVSUHPB. Сценарий \texttt{7fac4231} с бинарными переменными}
		\begin{tabular}{ p{2.5cm} p{3.3cm} p{3.4cm} }
			\emph{Способ} & \emph{Полное время расчета, мин} & \emph{Верхняя граница решения, $ \times 10^{10} $} \\
			\hline\hline\\[-3.5mm]
			{CBC+DOH} & 27.00 & $ 1.157865 $ \\
			\hline
			SCIP+MC ($ a $) & 18.05 {\color{blue} $\ +33.15 $\%} & $ 1.153273 $ {\color{blue} $\ +0.40 $\%} \\
			\hline
			SCIP+MC ($ b $) & 18.07 {\color{blue} $\ +33.07 $\%} & $ 1.118743 $ {\color{blue} $\ +3.38 $\%} \\
		\end{tabular}\label{tab:7fac4231}
	\end{table}
}

\subsubsection{Сценарий \texttt{50197DF7} с бинарными переменными}

\textbf{Статистика}\vspace*{1mm}

Общее количество переменных: 720954

Количество целочисленных переменных: 159332

Количество бинарных переменных: 2490

Количество ограничений: 600777

lp-файл: \url{https://disk.yandex.ru/d/qWeSKb2WEs6kQA}

\vspace*{5mm}\textbf{Анализ решения}\vspace*{1mm}

Пул решений задачи был найден с помощью следующих первичных эвристик:
\begin{itemize}
	\item INTSHIFTING,
	
	\item ONEOPT,
	
	\item GINS.
\end{itemize}

Фрагмент лога сессии SCIP
\begin{lstlisting}[
	style = bash,
	numbers = none
	]
	...
 time | node  | left  |LP iter|LP it/n|mem/heur|mdpt |vars |cons |rows |cuts |sepa|confs|strbr|  dualbound   | primalbound  |  gap   | compl. 
r 836s|   963 |   948 |155676 |  53.5 |intshift| 409 |  41k|  34k|  35k|4367 |  1 |  69 |7354 | 3.554610e+10 | 3.676991e+10 |   3.44%| unknown
i 836s|   964 |   947 |155676 |  53.5 |  oneopt| 409 |  41k|  34k|  35k|4367 |  0 |  69 |7354 | 3.554610e+10 | 3.676497e+10 |   3.43%| unknown
846s|  1000 |   985 |157559 |  53.4 |  2577M | 409 |  41k|  34k|  35k|4396 |  1 |  69 |7444 | 3.554610e+10 | 3.676497e+10 |   3.43%| unknown
L 885s|  1064 |  1049 |157869 |  50.5 |    gins| 409 |  41k|  34k|  35k|4397 |  1 |  69 |7484 | 3.554610e+10 | 3.659894e+10 |   2.96%| unknown
L 931s|  1064 |  1049 |157869 |  50.5 |    gins| 409 |  41k|  34k|  35k|4397 |  1 |  69 |7484 | 3.554610e+10 | 3.656967e+10 |   2.88%| unknown
i 962s|  1064 |  1049 |161589 |  54.0 |  oneopt| 409 |  41k|  34k|  35k|4397 |  1 |  69 |7484 | 3.554610e+10 | 3.656967e+10 |   2.88%| unknown
969s|  1100 |  1085 |161769 |  52.4 |  2620M | 409 |  41k|  34k|  35k|4397 |  1 |  69 |7532 | 3.554610e+10 | 3.656967e+10 |   2.88%| unknown
L 988s|  1164 |  1149 |161992 |  49.7 |    gins| 409 |  41k|  34k|  35k|4397 |  1 |  69 |7557 | 3.554610e+10 | 3.630031e+10 |   2.12%| unknown
L 993s|  1164 |  1149 |161992 |  49.7 |    gins| 409 |  41k|  34k|  35k|4397 |  1 |  69 |7557 | 3.554610e+10 | 3.625804e+10 |   2.00%| unknown
L1000s|  1164 |  1149 |161992 |  49.7 |    gins| 409 |  41k|  34k|  35k|4397 |  1 |  69 |7557 | 3.554610e+10 | 3.623675e+10 |   1.94%| unknown
\end{lstlisting}

Файл решения задачи доступен по ссылке \url{https://disk.yandex.ru/d/2_FDqS70q0UBqA}

Файл статистической сводки доступен по ссылке \url{https://disk.yandex.ru/d/SkRLoRYzQDI-Aw}

\vspace*{3mm}
\textbf{Вывод по сценарию}: описанная выше метаконфигурация приводит к решению задачи, которое оказывается по отношению к результату на доменно-ориентированных эвристиках для последнего решения из пула допустимых целочисленных решений на 2.87\% лучше в смысле целевой функции и на 36.08\% -- в смысле временных издержек (\tblref{tab:50197df7}).

В \tblref{tab:50197df7}  через SCIP+MC~$ (a) $ обозначается решение, построенное на метаконфигурации SCIP, отвечающее \emph{первому} допустимому целочисленному решению, верхняя граница которого не превышает верхнюю границу решения на доменно-ориентированных эвристиках, а через SCIP+MC~$ (b) $ -- решение, отвечающее \emph{последнему} допустимому целочисленному решению в наборе полученных.

Синим цветом обозначен выигрыш в процентах.

{
	%\rowcolors{2}{white}{lightgray!15}
	\begin{table}[!h]
		\centering
		\caption{Сводка результатов анализа эффективности\\метаконфигурации FZBIVSUHPB. Сценарий \texttt{50197df7} с бинарными переменными}
		\begin{tabular}{ p{2.5cm} p{3.3cm} p{3.4cm} }
			\emph{Способ} & \emph{Полное время расчета, мин} & \emph{Верхняя граница решения, $ \times 10^{10} $} \\
			\hline\hline\\[-3.5mm]
			{CBC+DOH} & 28.27 & $ 3.730552 $ \\
			\hline
			SCIP+MC ($ a $) & 13.93 {\color{blue} $\ +50.73 $\%} & $ 3.676991 $ {\color{blue} $\ +1.44 $\%} \\
			\hline
			SCIP+MC ($ b $) & 18.07 {\color{blue} $\ +36.08 $\%} & $ 3.623675 $ {\color{blue} $\ +2.87 $\%} \\
		\end{tabular}\label{tab:50197df7}
	\end{table}
}

\subsubsection{Сценарий \texttt{F398266B} с бинарными переменными}\label{sec:f398266b_bin}

\textbf{Статистика}\vspace*{1mm}

Общее количество переменных: 777271

Количество целочисленных переменных: 172449

Количество бинарных переменных: 2370

Количество ограничений: 655003

lp-файл: \url{https://disk.yandex.ru/d/4YFYJSB1I1wsmQ} 

\vspace*{5mm}\textbf{Анализ решения}\vspace*{1mm}

Пул решений задачи был найден с помощью следующих первичных эвристик:
\begin{itemize}
	\item DISTRIBUTIOINDIVING,
	
	\item ONEOPT,
	
	\item CROSSOVER.
\end{itemize}

Фрагмент лога сессии SCIP
\begin{lstlisting}[
	style = bash,
	numbers = none
	]
	...
 time | node  | left  |LP iter|LP it/n|mem/heur|mdpt |vars |cons |rows |cuts |sepa|confs|strbr|  dualbound   | primalbound  |  gap   | compl. 
d1163s|   433 |   434 |462507 | 790.8 |distribu|  51 |  59k|  48k|  49k|   0 |  1 |  17 |1387 | 5.857793e+10 | 6.054807e+10 |   3.36%| unknown
d1164s|   433 |   434 |462644 | 791.1 |distribu|  51 |  59k|  48k|  49k|   0 |  1 |  17 |1387 | 5.857793e+10 | 6.054779e+10 |   3.36%| unknown
d1164s|   433 |   434 |462746 | 791.3 |distribu|  51 |  59k|  48k|  49k|   0 |  1 |  17 |1387 | 5.857793e+10 | 6.054778e+10 |   3.36%| unknown
d1164s|   433 |   434 |462780 | 791.4 |distribu|  51 |  59k|  48k|  49k|   0 |  1 |  17 |1387 | 5.857793e+10 | 6.054776e+10 |   3.36%| unknown
d1164s|   433 |   434 |462801 | 791.4 |distribu|  51 |  59k|  48k|  49k|   0 |  1 |  17 |1387 | 5.857793e+10 | 6.054776e+10 |   3.36%| unknown
d1165s|   433 |   434 |462836 | 791.5 |distribu|  51 |  59k|  48k|  49k|   0 |  1 |  17 |1387 | 5.857793e+10 | 6.054776e+10 |   3.36%| unknown
d1165s|   433 |   434 |462856 | 791.6 |distribu|  51 |  59k|  48k|  49k|   0 |  1 |  17 |1387 | 5.857793e+10 | 6.054774e+10 |   3.36%| unknown
i1167s|   434 |   433 |463020 | 790.1 |  oneopt|  51 |  59k|  48k|  49k|4333 |  0 |  17 |1387 | 5.857793e+10 | 6.053918e+10 |   3.35%| unknown
1250s|   500 |   501 |531180 | 822.2 |  3321M |  51 |  59k|  48k|  49k|4529 |  1 |  26 |1402 | 5.857793e+10 | 6.053918e+10 |   3.35%| unknown
1579s|   600 |   601 |663342 | 905.6 |  3398M |  51 |  59k|  48k|  49k|5175 |  1 |  36 |1426 | 5.857932e+10 | 6.053918e+10 |   3.35%| unknown
L1892s|   634 |   635 |704819 | 922.5 |crossove|  55 |  59k|  48k|  49k|5448 |  2 |  41 |1433 | 5.858028e+10 | 6.021605e+10 |   2.79%| unknown
i1895s|   634 |   635 |715376 | 939.1 |  oneopt|  55 |  59k|  48k|  49k|5448 |  2 |  41 |1433 | 5.858028e+10 | 6.021603e+10 |   2.79%| unknown
1952s|   700 |   701 |770566 | 929.4 |  3457M |  63 |  59k|  48k|  49k|5644 |  1 |  50 |1442 | 5.858050e+10 | 6.021603e+10 |   2.79%| unknown
2095s|   800 |   801 |879949 | 950.0 |  3489M |  65 |  59k|  48k|  49k|5964 |  1 |  62 |1476 | 5.858065e+10 | 6.021603e+10 |   2.79%| unknown
\end{lstlisting}

Файл решения задачи доступен по ссылке \url{https://disk.yandex.ru/d/KXzdrUx6TZbXEw}

Файл статистической сводки доступен по ссылке \url{https://disk.yandex.ru/d/FERoaFsr5zbkjA}

\vspace*{3mm}
\textbf{Вывод по сценарию}: описанная выше метаконфигурация приводит к решению задачи, которое оказывается по отношению к результату на доменно-ориентированных эвристиках для последнего решения из пула допустимых целочисленных решений на 0.97\% лучше в смысле целевой функции и на 56.24\% -- в смысле временных издержек (\tblref{tab:f398266b}).

В \tblref{tab:f398266b}  через SCIP+MC~$ (a) $ обозначается решение, построенное на метаконфигурации SCIP, отвечающее \emph{первому} допустимому целочисленному решению, верхняя граница которого не превышает верхнюю границу решения на доменно-ориентированных эвристиках, а через SCIP+MC~$ (b) $ -- решение, отвечающее \emph{последнему} допустимому целочисленному решению в наборе полученных.

Синим цветом обозначен выигрыш в процентах.

{
	%\rowcolors{2}{white}{lightgray!15}
	\begin{table}[!h]
		\centering
		\caption{Сводка результатов анализа эффективности\\метаконфигурации FZBIVSUHPB. Сценарий \texttt{f398266b} с бинарными переменными}
		\begin{tabular}{ p{2.5cm} p{3.3cm} p{3.4cm} }
			\emph{Способ} & \emph{Полное время расчета, мин} & \emph{Верхняя граница решения, $ \times 10^{10} $} \\
			\hline\hline\\[-3.5mm]
			{CBC+DOH} & 72.17 & $ 6.080841 $ \\
			\hline
			SCIP+MC ($ a $) & 19.38 {\color{blue} $\ +73.15 $\%} & $ 6.054807 $ {\color{blue} $\ +0.43 $\%} \\
			\hline
			SCIP+MC ($ b $) & 31.58 {\color{blue} $\ +56.24 $\%} & $ 6.021603 $ {\color{blue} $\ +0.97 $\%} \\
		\end{tabular}\label{tab:f398266b}
	\end{table}
}

\subsubsection{Сценарий \texttt{337} с бинарными переменными}

\textbf{Статистика}\vspace*{1mm}

Общее количество переменных: 859230

Количество целочисленных переменных: 173622

Количество бинарных переменных: 155

Количество ограничений: 624637

lp-файл: \url{https://disk.yandex.ru/d/Kc11p9v7D-kxYA}

\vspace*{5mm}\textbf{Анализ решения}\vspace*{1mm}

Пул решений задачи был найден с помощью следующих первичных эвристик:
\begin{itemize}
	\item INTSHIFTING,
	
	\item RENS,
	
	\item ONEOPT.
\end{itemize}

Фрагмент лога сессии SCIP

\begin{lstlisting}[
style = bash,
numbers = none
]
time | node  | left  |LP iter|LP it/n|mem/heur|mdpt |vars |cons |rows |cuts |sepa|confs|strbr|  dualbound   | primalbound  |  gap   | compl. 
r 107s|     1 |     0 | 55407 |     - |intshift|   0 |  56k|  43k|  45k|1799 | 13 |   0 |   0 | 2.947544e+10 | 4.344720e+10 |  47.40%| unknown
L 247s|     1 |     0 | 55407 |     - |    rens|   0 |  56k|  43k|  45k|1799 | 13 |   0 |   0 | 2.947544e+10 | 3.022206e+10 |   2.53%| unknown
249s|     1 |     0 | 55407 |     - |  2785M |   0 |  56k|  43k|  45k|1799 | 13 |   0 |   0 | 2.947544e+10 | 3.022206e+10 |   2.53%| unknown
i 250s|     1 |     0 | 58839 |     - |  oneopt|   0 |  56k|  43k|  45k|1799 | 13 |   0 |   0 | 2.947544e+10 | 3.022205e+10 |   2.53%| unknown
250s|     1 |     0 | 58839 |     - |  2809M |   0 |  56k|  43k|  45k|1799 | 13 |   0 |   0 | 2.947544e+10 | 3.022205e+10 |   2.53%| unknown
251s|     1 |     0 | 58891 |     - |  2813M |   0 |  56k|  43k|  45k|1820 | 14 |   0 |   0 | 2.947544e+10 | 3.022205e+10 |   2.53%| unknown
251s|     1 |     0 | 58900 |     - |  2813M |   0 |  56k|  43k|  44k|1824 | 15 |   0 |   0 | 2.947544e+10 | 3.022205e+10 |   2.53%| unknown
253s|     1 |     0 | 59074 |     - |  2816M |   0 |  56k|  43k|  44k|1824 | 15 |   0 |  12 | 2.947544e+10 | 3.022205e+10 |   2.53%| unknown
254s|     1 |     0 | 59236 |     - |  2821M |   0 |  56k|  43k|  44k|1918 | 16 |   0 |  12 | 2.948327e+10 | 3.022205e+10 |   2.51%| unknown
254s|     1 |     0 | 59300 |     - |  2821M |   0 |  56k|  43k|  44k|1945 | 17 |   0 |  12 | 2.948327e+10 | 3.022205e+10 |   2.51%| unknown
255s|     1 |     0 | 59321 |     - |  2821M |   0 |  56k|  43k|  44k|1945 | 17 |   0 |  19 | 2.948327e+10 | 3.022205e+10 |   2.51%| unknown
256s|     1 |     0 | 59349 |     - |  2825M |   0 |  56k|  43k|  44k|1959 | 18 |   0 |  19 | 2.948327e+10 | 3.022205e+10 |   2.51%| unknown
256s|     1 |     0 | 59352 |     - |  2825M |   0 |  56k|  43k|  44k|1964 | 19 |   0 |  19 | 2.948327e+10 | 3.022205e+10 |   2.51%| unknown
258s|     1 |     0 | 59368 |     - |  2825M |   0 |  56k|  43k|  44k|1964 | 19 |   0 |  35 | 2.957927e+10 | 3.022205e+10 |   2.17%| unknown
259s|     1 |     0 | 59451 |     - |  2829M |   0 |  56k|  43k|  44k|2014 | 20 |   0 |  35 | 2.957927e+10 | 3.022205e+10 |   2.17%| unknown
259s|     1 |     0 | 59466 |     - |  2829M |   0 |  56k|  43k|  44k|2024 | 21 |   0 |  35 | 2.957927e+10 | 3.022205e+10 |   2.17%| unknown
259s|     1 |     2 | 59466 |     - |  2829M |   0 |  56k|  43k|  44k|2024 | 21 |   0 |  35 | 2.957927e+10 | 3.022205e+10 |   2.17%| unknown
\end{lstlisting}

Файл решения задачи доступен по ссылке \url{https://disk.yandex.ru/d/zwVhKYKEMlMlQw}

Файл статистической сводки доступен по ссылке \url{https://disk.yandex.ru/d/T9sAbRH6uWh4Uw}

\vspace*{3mm}
\textbf{Вывод по сценарию}: описанная выше метаконфигурация приводит к решению задачи, которое оказывается по отношению к результату на доменно-ориентированных эвристиках для последнего решения из пула допустимых целочисленных решений на ...\% лучше в смысле целевой функции и на ...\% -- в смысле временных издержек (\tblref{tab:337}).

В \tblref{tab:337}  через SCIP+MC~$ (a) $ обозначается решение, построенное на метаконфигурации SCIP, отвечающее \emph{первому} допустимому целочисленному решению, верхняя граница которого не превышает верхнюю границу решения на доменно-ориентированных эвристиках, а через SCIP+MC~$ (b) $ -- решение, отвечающее \emph{последнему} допустимому целочисленному решению в наборе полученных.

Синим цветом обозначен выигрыш в процентах.

{
	%\rowcolors{2}{white}{lightgray!15}
	\begin{table}[!h]
		\centering
		\caption{Сводка результатов анализа эффективности\\метаконфигурации FZBIVSUHPB. Сценарий \texttt{337} с бинарными переменными}
		\begin{tabular}{ p{2.5cm} p{3.3cm} p{3.4cm} }
			\emph{Способ} & \emph{Полное время расчета, мин} & \emph{Верхняя граница решения, $ \times 10^{10} $} \\
			\hline\hline\\[-3.5mm]
			{CBC+DOH} & 18.00 & $ 4.047865 $ \\
			\hline
			SCIP+MC ($ a $) & 4.12 {\color{blue} $\ +77.11 $\%} & $ 3.022206 $ {\color{blue} $\ +25.34 $\%} \\
			\hline
			SCIP+MC ($ b $) &  4.30 {\color{blue} $\ +76.11 $\%} & $ 3.022205 $ {\color{blue} $\ +25.34 $\%} \\
		\end{tabular}\label{tab:337}
	\end{table}







%\begin{landscape}
	\begin{figure}[!h]
		\centering
		\includegraphics[scale=0.42]{figures/a78cbead_autorestartnodes_1_2_phase.pdf}
		\caption{ Динамика изменения верхней границы решения и числа конфликтов во времени в зависимости \\от значения параметра \texttt{autorestartnodes}. Сценарий \texttt{input\_a78cbead}. Первая и вторая фазы поиска решения }\label{fig:a78cbead_autorestartnodes_1_2_phase}
	\end{figure}
%\end{landscape}

%\begin{landscape}
\begin{figure}[!h]
	\centering
	\includegraphics[scale=0.42]{figures/a78cbead_autorestartnodes_3_phase.pdf}
	\caption{ Динамика изменения верхней границы решения и числа конфликтов во времени в зависимости \\от значения параметра \texttt{autorestartnodes}. Сценарий \texttt{a78cbead}. Третья фаза поиска решения }\label{fig:a78cbead_autorestartnodes_3_phase}
\end{figure}
%\end{landscape}

%\begin{landscape}
\begin{figure}[!h]
	\centering
	\includegraphics[scale=0.42]{figures/50197df7_autorestartnodes.pdf}
	\caption{ Динамика изменения верхней границы решения и числа конфликтов во времени в зависимости \\от значения параметра \texttt{autorestartnodes}. Сценарий \texttt{50197df7}. Третья фаза поиска решения}\label{fig:50197df7_autorestartnodes}
\end{figure}
%\end{landscape}

%\begin{landscape}
\begin{figure}[!h]
	\centering
	\includegraphics[scale=0.42]{figures/7fac4231_autorestartnodes.pdf}
	\caption{ Динамика изменения верхней границы решения и числа конфликтов во времени в зависимости \\от значения параметра \texttt{autorestartnodes}. Сценарий \texttt{7fac4231}. Третья фаза поиска решения}\label{fig:7fac4231_autorestartnodes}
\end{figure}
%\end{landscape}

\subsection{Поиск решения на базе методов машинного и глубокого обучения}

Условимся \emph{сценарием обучающего поднабора} называть сценарий (математическую постановку задачи, описанную в терманах математического программирования) из коллекции сценариев, которые используются на \emph{обучающей фазе} модели машинного обучения.

\emph{Сценарием тестового поднабора} условимся называть сценарий, который используется \emph{для построения прогноза} с помощью модели машинного обучения.

\subsubsection{Простое декартово произведение сценариев \emph{с} бинарными переменными}

 Рассмотрим \emph{некоммутативные} пары вида <<сценарий обучающего поднабора -- сценарий тестового поднабора>> подгруппы сценариев с бинарными переменными (см. раздел \ref{sec:ikp_bins}):
\begin{itemize}
	\item \texttt{7fac4231\_bin.lp},
	
	\item \texttt{a78cbead\_bin.lp},
	
	\item \texttt{f398266b\_bin.lp},
	
	\item \texttt{50197df7\_bin.lp},
	
	\item \texttt{337\_bin.lp}.
\end{itemize}

Если коллекция сценариев содержит $ n $ сценариев, то существует $ n (n - 1) $ возможных некоммутативных пар.

\paragraph{обучение на сценарии \texttt{7fac4231\_bin.lp}, тестирование на сценарии \texttt{50197df7\_bin.lp}}

...

\paragraph{обучение на сценарии \texttt{a78cbead\_bin.lp}, тестирование на сценарии \texttt{50197df7\_bin.lp}}

...


\paragraph{обучение на сценарии \texttt{f398266b\_bin.lp}, тестирование на сценарии \texttt{50197df7\_bin.lp}}

...


\paragraph{обучение на сценарии \texttt{337\_bin.lp}, тестирование на сценарии \texttt{50197df7\_bin.lp}}

...провал

\paragraph{обучение на сценарии \texttt{7fac4231\_bin.lp}, тестирование на сценарии \texttt{50197df7\_bin.lp}}



\section{Описание вычислительных экспериментов на сценариях группы MBO}

\section{Описание вычислительных экспериментов \\на сценариях MIPLIB~2017}

\subsection{Сценарии со статусом <<open>>}

\subsubsection{Сценарий \texttt{DLR2}}

\url{https://miplib.zib.de/WebData/instances/dlr2.mps.gz}

\subsubsection{Сценарий \texttt{CVRPA-N64K9VRPI}}

\url{https://miplib.zib.de/WebData/instances/cvrpa-n64k9vrpi.mps.gz}

\subsection{Сценарии со статусом <<hard>>}

\subsubsection{Сценарий \texttt{CRYPTANALYSISKB128N5OBJ14}}

\url{https://miplib.zib.de/WebData/instances/cryptanalysiskb128n5obj14.mps.gz}

\subsection{Сценарии со статусом <<easy>>}

\subsubsection{Сценарий \texttt{NEOS-4332801-seret}}

\url{https://miplib.zib.de/WebData/instances/neos-4332801-seret.mps.gz}


\newpage
\listoffigures\addcontentsline{toc}{section}{Список иллюстраций}

\listoftables\addcontentsline{toc}{section}{Список таблиц}

% Источники в "Газовой промышленности" нумеруются по мере упоминания 
\begin{thebibliography}{99}\addcontentsline{toc}{section}{Список литературы}
	\bibitem{ivanov:rl-2022}{\emph{Иванов} Конспект по обучению с подкреплением, 2022}
	
	\bibitem{geron:ml-2018}{\emph{Жерон, О.} Прикладное машинное обучение с помощью Scikit-Learn и TensorFlow: концепции, инструменты и техники для создания интеллектуальных систем. -- СПб.: ООО <<Альфа-книга>>, 2018. -- 688 с.}
	
	\bibitem{soenen:effect-hyper-param-tuning:2021}{\emph{Soenen J. etc.} The Effect of Hyperparameter Tuning on the Comparative Evaluation of Unsupervised Anomaly Detection Methods, 2021}
\end{thebibliography}

\end{document}
